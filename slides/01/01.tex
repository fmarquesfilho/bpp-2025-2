\documentclass[10pt, aspectratio=169]{beamer}
\usepackage[utf8]{inputenc}
\usepackage[portuguese]{babel}
\usepackage{graphicx}
\usepackage{ragged2e}
\usepackage{hyperref}
\usepackage{booktabs}
\usepackage{tabularx}
\usepackage{colortbl}
\usepackage{xcolor}

\usetheme{Madrid}
\usecolortheme{seahorse}

\title{Boas Práticas de Programação}
\subtitle{Plano do Curso -- 2025.2}
\author{Prof. Fernando Figueira}
\institute{DIMAp -- UFRN}
\date{\today}

\setbeamertemplate{footline}[frame number]
\setbeamertemplate{navigation symbols}{}

\definecolor{azulUni}{RGB}{0, 51, 102}
\setbeamercolor{palette primary}{bg=azulUni,fg=white}
\setbeamercolor{palette secondary}{bg=azulUni!80!black,fg=white}
\setbeamercolor{palette tertiary}{bg=azulUni!60!black,fg=white}
\setbeamercolor{palette quaternary}{bg=azulUni!40!black,fg=white}
\setbeamercolor{structure}{fg=azulUni}
\setbeamercolor{section in toc}{fg=azulUni}
\setbeamercolor{subsection in toc}{fg=azulUni!80!black}

\begin{document}

\begin{frame}
\titlepage
\end{frame}

\begin{frame}{Sumário}
\tableofcontents
\end{frame}

\section{Informações Gerais}
\begin{frame}{Informações Gerais}
\begin{itemize}
    \item \textbf{Período:} 2025.2
    \item \textbf{Dias de aula:} Sextas-feiras
    \item \textbf{Modalidade:} Presencial (exceto acompanhamento de projeto)
    \item \textbf{Linguagem:} Livre para o projeto
    \item \textbf{Repositório:} \url{https://github.com/fmarquesfilho/bpp-2025-2}
\end{itemize}
\end{frame}

\section{Ementa e Abordagem}
\begin{frame}{Ementa do Curso}
\begin{block}{Conteúdos abordados}
\begin{itemize}
    \item Organização de código em classes e pacotes
    \item Critérios de qualidade de rotinas
    \item Programação defensiva e com pseudo-código
    \item Padrões de comentários e nomenclatura
    \item Estruturas de dados e controle de código
    \item Automação de compilação e testes
    \item Análise de cobertura e desempenho
    \item Técnicas de depuração
\end{itemize}
\end{block}
\end{frame}

\begin{frame}{Abordagem Pedagógica}
\begin{block}{Foco na prática}
\begin{itemize}
    \item Princípios de \textbf{código limpo} e \textbf{SOLID}
    \item Identificação de \textbf{code smells}
    \item Técnicas de \textbf{refatoração}
    \item Ferramentas open-source de análise estática
    \item Aplicação em projetos reais
\end{itemize}
\end{block}

\begin{block}{Recursos principais}
\begin{itemize}
    \item \href{https://luzkan.github.io/smells/}{Catálogo de Code Smells e Refatorações}
    \item Exemplos práticos da internet
    \item Mineração de repositórios
\end{itemize}
\end{block}
\end{frame}

\section{Metodologia de Avaliação}
\begin{frame}{Sistema de Avaliação}
\begin{table}
\centering
\begin{tabular}{lll}
\toprule
\textbf{Unidade} & \textbf{Tipo} & \textbf{Descrição} \\
\midrule
U1 (10,0) & Trabalho Prático & Planejamento do projeto \\
U2 (10,0) & Prova Escrita & Conceitos teóricos e práticos \\
U3 (10,0) & Projeto Final & Desenvolvimento do MVP \\
\bottomrule
\end{tabular}
\end{table}

\begin{block}{Nota Final}
Média aritmética das três unidades
\end{block}
\end{frame}

\begin{frame}{Projeto do Curso}
\begin{columns}
\begin{column}{0.6\textwidth}
\begin{block}{Características}
\begin{itemize}
    \item Individual ou grupos de 2-3 pessoas
    \item Modelo MVP (Minimum Viable Product)
    \item Entregas com vídeo (5-8 min)
    \item Artefatos em PDF/ZIP
    \item Foco na aplicação prática
\end{itemize}
\end{block}
\end{column}
\end{columns}
\end{frame}

\section{Calendário do Curso}
\begin{frame}{Primeiras Semanas}
\begin{table}
\scriptsize
\centering
\begin{tabular}{ll}
\toprule
\textbf{Data} & \textbf{Atividade} \\
\midrule
22/08/25 & Apresentação do curso \\
29/08/25 & Código Limpo: Nomenclatura e Estrutura \\
05/09/25 & Code Smells e Ferramentas de Detecção \\
12/09/25 & Princípios SOLID \\
19/09/25 & Técnicas de Refatoração \\
26/09/25 & Ferramentas de Análise Estática \\
\bottomrule
\end{tabular}
\end{table}
\end{frame}

\begin{frame}{Entregas e Avaliações}
\begin{table}
\scriptsize
\centering
\begin{tabular}{ll}
\toprule
\textbf{Data} & \textbf{Atividade} \\
\midrule
02/10/25 & \textbf{Entrega U1} (via SIGAA até 23:59) \\
03/10/25 & \textcolor{red}{Feriado - Não haverá aula} \\
10/10/25 & Mineração de Repositórios I \\
17/10/25 & Mineração de Repositórios II \\
24/10/25 & Programação Defensiva \\
31/10/25 & Automação de Testes \\
07/11/25 & \textbf{Prova U2} (Avaliação Escrita) \\
\bottomrule
\end{tabular}
\end{table}
\end{frame}

\begin{frame}{Semanas Finais}
\begin{table}
\scriptsize
\centering
\begin{tabular}{ll}
\toprule
\textbf{Data} & \textbf{Atividade} \\
\midrule
14/11/25 & Depuração e Análise de Desempenho \\
21/11/25 & \textcolor{red}{Feriado - Não haverá aula} \\
28/11/25 & Acompanhamento do Projeto (online) \\
05/12/25 & \textbf{Entrega U3} (via SIGAA até 23:59) \\
12/12/25 & Recuperação \\
\bottomrule
\end{tabular}
\end{table}
\end{frame}

\section{Datas Importantes}
\begin{frame}{Feriados e Suspensões}
\begin{alertblock}{Datas sem aula}
\begin{itemize}
    \item \textbf{03/10/25} - Dia dos Mártires de Cunhaú e Uruaçu
    \item \textbf{28/10/25} - Dia do Servidor Público
    \item \textbf{21/11/25} - Feriado Municipal em Natal
\end{itemize}
\end{alertblock}

\begin{block}{Observações}
\begin{itemize}
    \item Acompanhamento do projeto (28/11) será online
    \item Plantão para dúvidas na entrega final (05/12)
\end{itemize}
\end{block}
\end{frame}

\section{Recursos e Materiais}
\begin{frame}{Materiais de Apoio}
\begin{block}{Principais recursos}
\begin{itemize}
    \item \textbf{Livro:} ``Clean Code'' de Robert C. Martin
    \item \textbf{Catálogo:} \url{https://luzkan.github.io/smells/}
    \item \textbf{Ferramentas:} SonarQube, PMD, Checkstyle
    \item \textbf{Testes:} JUnit, Valgrind, outras ferramentas
    \item \textbf{Repositório:} GitHub do curso
\end{itemize}
\end{block}
\end{frame}

\begin{frame}{Ferramentas Utilizadas}
\begin{columns}
\begin{column}{0.5\textwidth}
\begin{block}{Análise Estática}
\begin{itemize}
    \item SonarQube
    \item PMD
    \item Checkstyle
    \item ESLint/TSLint
\end{itemize}
\end{block}
\end{column}
\begin{column}{0.5\textwidth}
\begin{block}{Testes e Qualidade}
\begin{itemize}
    \item JUnit/TestNG
    \item Jest/Mocha
    \item JaCoCo
    \item Valgrind
\end{itemize}
\end{block}
\end{column}
\end{columns}
\end{frame}

\section{Contato e Informações}
\begin{frame}{Comunicação e Contato}
\begin{block}{Canais oficiais}
\begin{itemize}
    \item \textbf{SIGAA:} Comunicações oficiais e entregas
    \item \textbf{Repositório GitHub:} \url{https://github.com/fmarquesfilho/bpp-2025-2}
    \item \textbf{E-mail:} [seu.email@universidade.edu]
\end{itemize}
\end{block}

\begin{block}{Modalidades de aula}
\begin{itemize}
    \item \textbf{Presencial:} Aulas regulares
    \item \textbf{Online:} Acompanhamento de projeto (28/11)
    \item \textbf{Plantão:} Dúvidas sobre entregas (05/12)
\end{itemize}
\end{block}
\end{frame}

\begin{frame}{Expectativas de Aprendizado}
\begin{block}{Ao final do curso, você será capaz de:}
\begin{itemize}
    \item Identificar e corrigir code smells
    \item Aplicar princípios SOLID e Clean Code
    \item Utilizar ferramentas de análise estática
    \item Desenvolver código de alta qualidade
    \item Implementar testes automatizados
    \item Realizar refatorações eficazes
    \item Analisar desempenho e detectar gargalos
\end{itemize}
\end{block}
\end{frame}

\begin{frame}
\centering
\Huge \textbf{Bons Estudos!}
\vspace{1cm}
\\
\large Dúvidas e sugestões são sempre bem-vindas!
\\
\vspace{0.5cm}
\small \url{https://github.com/fmarquesfilho/bpp-2025-2}
\end{frame}

\end{document}
