\documentclass[aspectratio=169]{beamer}
\usepackage[utf8]{inputenc}
\usepackage[T1]{fontenc}
\usepackage[brazilian]{babel}
\usepackage{listings}
\usepackage{xcolor}
\usepackage{tikz}
\usepackage{fontawesome5}
\usetikzlibrary{shapes,arrows,positioning}

% Tema
\usetheme{Madrid}
\usecolortheme{default}

% Cores personalizadas
\definecolor{codecolor}{rgb}{0.95,0.95,0.95}
\definecolor{commentgreen}{rgb}{0,0.6,0}
\definecolor{keywordblue}{rgb}{0,0,0.8}

% Configuração de códig
\lstset{
    language=Python,
    basicstyle=\ttfamily\small,
    keywordstyle=\color{keywordblue}\bfseries,
    commentstyle=\color{commentgreen},
    stringstyle=\color{red},
    backgroundcolor=\color{codecolor},
    frame=single,
    breaklines=true,
    showstringspaces=false,
    tabsize=2
}

% Informações do documento
\title{Automação de Testes e Análise de Cobertura}
\subtitle{Boas Práticas de Programação - 2025.2}
\author{Prof. [Fernando Figueira]}
\institute{Universidade Federal do Rio Grande do Norte}
\date{07 de Novembro de 2025}

\begin{document}

% Slide de título
\frame{\titlepage}

% Sumário
\begin{frame}{Agenda}
    \tableofcontents
\end{frame}

%=============================================================================
\section{Introdução à Automação de Testes}
%=============================================================================

\begin{frame}{Por que Automatizar Testes?}
    \begin{columns}
        \begin{column}{0.5\textwidth}
            \textbf{Benefícios:}
            \begin{itemize}
                \item \faCheckCircle\ Detecção precoce de bugs
                \item \faCheckCircle\ Redução de custos
                \item \faCheckCircle\ Documentação viva
                \item \faCheckCircle\ Confiança para refatorar
                \item \faCheckCircle\ Integração com CI/CD
            \end{itemize}
        \end{column}
        \begin{column}{0.5\textwidth}
            \begin{alertblock}{Estatística}
                Bugs encontrados na produção custam \textbf{15x mais} para corrigir do que bugs encontrados durante o desenvolvimento.
            \end{alertblock}
        \end{column}
    \end{columns}
\end{frame}

\begin{frame}{A Pirâmide de Testes}
    \begin{center}
        \vspace{3em}
        \begin{tikzpicture}[scale=0.8, every node/.style={transform shape}]
            % UI Tests (End-to-End) i'd like the pyramid to be divided into three sections with labels and percentages on the right side, right now the top edge is flattened

            \draw[fill=red!30] (-1,10) -- (1,10) -- (2,8) -- (-2,8) -- cycle;
            \node at (0,9) {\textbf{End-to-End}};
            \node[align=center, font=\small] at (0,8.5) {Lentos\\Caro};
            
            % Integration Tests
            \draw[fill=yellow!40] (-2,8) -- (2,8) -- (3,6) -- (-3,6) -- cycle;
            \node at (0,7) {\textbf{Integração}};
            \node[font=\small] at (0,6.5) {Moderados};
            
            % Unit Tests
            \draw[fill=green!30] (-3,6) -- (3,6) -- (4.5,4) -- (-4.5,4) -- cycle;
            \node at (0,5.2) {\textbf{Testes Unitários}};
            \node[align=center, font=\small] at (0,4.5) {Rápido\\Barato};
            
            % Percentages
            \node[right, font=\small] at (0,9.5) {10\%};
            \node[right, font=\small] at (3.0,6.5) {20\%};
            \node[right, font=\small] at (5,4.7) {70\%};
        \end{tikzpicture}
    \end{center}
\end{frame}

\begin{frame}{Tipos de Testes}
    \begin{block}{Testes Unitários (70\%)}
        Testam \textbf{funções e métodos isolados}. Rápidos e focados.
    \end{block}
    
    \begin{block}{Testes de Integração (20\%)}
        Testam \textbf{interação entre componentes}. Banco de dados, APIs, módulos.
    \end{block}
    
    \begin{block}{Testes End-to-End (10\%)}
        Testam \textbf{fluxo completo do usuário}. Interface gráfica, simulação real.
    \end{block}
\end{frame}

\begin{frame}[fragile]{Características de Bons Testes: FIRST}
    \begin{description}
        \item[\textbf{F}ast] Executam rapidamente (segundos, não minutos)
        \item[\textbf{I}ndependent] Não dependem de outros testes
        \item[\textbf{R}epeatable] Resultados consistentes em qualquer ambiente
        \item[\textbf{S}elf-validating] Pass/Fail claro, sem verificação manual
        \item[\textbf{T}imely] Escritos junto com o código (idealmente antes!)
    \end{description}
    
    \vspace{1em}
    \begin{alertblock}{Atenção}
        Um teste que não segue FIRST é um teste frágil!
    \end{alertblock}
\end{frame}

%=============================================================================
\section{Análise de Cobertura de Código}
%=============================================================================

\begin{frame}{O que é Cobertura de Código?}
    \begin{block}{Definição}
        Percentual do código que é \textbf{executado} pelos testes automatizados.
    \end{block}
    
    \vspace{1em}
    
    \begin{columns}
        \begin{column}{0.5\textwidth}
            \textbf{O que medir:}
            \begin{itemize}
                \item Linhas executadas
                \item Branches (if/else)
                \item Funções chamadas
                \item Condições booleanas
            \end{itemize}
        \end{column}
        \begin{column}{0.5\textwidth}
            \begin{alertblock}{Importante}
                100\% de cobertura \\ 
                $\neq$ \\
                100\% de qualidade
            \end{alertblock}
        \end{column}
    \end{columns}
\end{frame}

\begin{frame}{Tipos de Cobertura}
    \begin{enumerate}
        \item \textbf{Cobertura de Linhas}
        \begin{itemize}
            \item Quantas linhas foram executadas?
            \item Meta: 70-80\% para código geral
        \end{itemize}
        
        \vspace{0.5em}
        
        \item \textbf{Cobertura de Branches}
        \begin{itemize}
            \item Quantos caminhos (if/else) foram testados?
            \item Meta: 60-70\% para código geral
        \end{itemize}
        
        \vspace{0.5em}
        
        \item \textbf{Cobertura de Funções}
        \begin{itemize}
            \item Quantas funções foram chamadas?
            \item Meta: 80-90\% para módulos críticos
        \end{itemize}
    \end{enumerate}
\end{frame}

\begin{frame}[fragile]{Exemplo: Código com Baixa Cobertura}
    \begin{lstlisting}
def calcular_desconto(preco, cupom):
    if cupom == "DESC10":
        return preco * 0.9
    elif cupom == "DESC20":
        return preco * 0.8
    elif cupom == "FRETE_GRATIS":
        return preco
    else:
        raise ValueError("Cupom invalido")
    \end{lstlisting}
    
    \vspace{0.5em}
    
    \begin{lstlisting}
def test_desconto():
    assert calcular_desconto(100, "DESC10") == 90
    \end{lstlisting}
    
    \begin{alertblock}{Problema}
        Cobertura de linhas: 37.5\% | Branches: 25\%
    \end{alertblock}
\end{frame}

\begin{frame}[fragile]{Exemplo: Código com Boa Cobertura}
    \begin{lstlisting}
def test_desconto_10():
    assert calcular_desconto(100, "DESC10") == 90

def test_desconto_20():
    assert calcular_desconto(100, "DESC20") == 80

def test_frete_gratis():
    assert calcular_desconto(100, "FRETE_GRATIS") == 100

def test_cupom_invalido():
    with pytest.raises(ValueError):
        calcular_desconto(100, "INVALIDO")
    \end{lstlisting}
    
    \begin{exampleblock}{Resultado}
        Cobertura de linhas: 100\% | Branches: 100\%
    \end{exampleblock}
\end{frame}

\begin{frame}{Métricas de Qualidade}
    \begin{center}
        \begin{tabular}{|l|c|c|}
            \hline
            \textbf{Categoria} & \textbf{Mínimo} & \textbf{Ideal} \\
            \hline
            Cobertura de Linhas & 70\% & 80-90\% \\
            \hline
            Cobertura de Branches & 60\% & 70-80\% \\
            \hline
            Módulos Críticos & 85\% & 95\%+ \\
            \hline
            Código Legado & 50\% & 70\% \\
            \hline
        \end{tabular}
    \end{center}
    
    \vspace{1em}
    
    \begin{block}{Boas Práticas}
        \begin{itemize}
            \item \faCheck\ Priorize código crítico e complexo
            \item \faCheck\ Use cobertura como guia, não como meta absoluta
            \item \faTimes\ Não infle cobertura com testes vazios
        \end{itemize}
    \end{block}
\end{frame}

%=============================================================================
\section{Ferramentas e Boas Práticas}
%=============================================================================

\begin{frame}{Ferramentas por Linguagem}
    \begin{center}
        \begin{tabular}{|l|l|l|}
            \hline
            \textbf{Linguagem} & \textbf{Framework} & \textbf{Cobertura} \\
            \hline
            Python & pytest, unittest & coverage.py \\
            \hline
            JavaScript & Jest, Mocha & Istanbul, nyc \\
            \hline
            Java & JUnit, TestNG & JaCoCo \\
            \hline
            C/C++ & Google Test, Catch2 & gcov, lcov \\
            \hline
            C\# & NUnit, xUnit & OpenCover \\
            \hline
            Go & testing (built-in) & go test -cover \\
            \hline
        \end{tabular}
    \end{center}
\end{frame}

\begin{frame}[fragile]{Padrão AAA (Arrange-Act-Assert)}
    \begin{block}{Estrutura Clara de Testes}
        Todo teste deve seguir três fases bem definidas:
    \end{block}
    
    \begin{lstlisting}
def test_calcular_desconto():
    # ARRANGE - Preparar dados e dependencias
    produto = Produto(nome="Notebook", preco=3000)
    desconto = 0.10
    
    # ACT - Executar a acao
    preco_final = produto.aplicar_desconto(desconto)
    
    # ASSERT - Verificar resultado
    assert preco_final == 2700
    assert produto.preco == 3000  # nao modifica original
    \end{lstlisting}
\end{frame}

\begin{frame}{O que Testar?}
    \begin{columns}
        \begin{column}{0.5\textwidth}
            \textcolor{green!70!black}{\faCheck\ \textbf{Priorize testar:}}
            \begin{itemize}
                \item Lógica de negócio
                \item Cálculos e transformações
                \item Validações
                \item Casos limite (edge cases)
                \item Tratamento de erros
                \item Funções complexas
            \end{itemize}
        \end{column}
        \begin{column}{0.5\textwidth}
            \textcolor{red}{\faTimes\ \textbf{Não priorize:}}
            \begin{itemize}
                \item Getters/setters simples
                \item Código de terceiros
                \item Constantes
                \item Configurações triviais
                \item Framework/biblioteca
            \end{itemize}
        \end{column}
    \end{columns}
\end{frame}

%=============================================================================
\section{Demonstração Prática}
%=============================================================================

\begin{frame}{Parte Prática}
    \begin{center}
        \Huge Vamos à Prática! \faLaptopCode
    \end{center}
    
    \vspace{2em}
    
    \Large
    \begin{itemize}
        \item Exercício 1: Testes Unitários
        \item Exercício 2: Cobertura de Código
        \item Exercício 3: TDD na Prática 
    \end{itemize}
    
    \vspace{1em}
    
    \normalsize
    \textbf{Ferramenta:} Replit (replit.com) ou Google Colab
\end{frame}

\begin{frame}[fragile]{Exercício 1: Sistema de Senhas}
    \textbf{Contexto:} Validação de senhas seguras
    
    \vspace{1em}
    
    \textbf{Sua tarefa:}
    \begin{itemize}
        \item Implementar testes para validação de senha
        \item Testar casos válidos e inválidos
        \item Testar cálculo de força da senha
    \end{itemize}
    
    \vspace{1em}
    
    \textbf{Requisitos:}
    \begin{itemize}
        \item Senha deve ter no mínimo 8 caracteres
        \item Deve conter maiúscula, minúscula e número
        \item Força: Fraca, Média ou Forte
    \end{itemize}
\end{frame}

\begin{frame}[fragile]{Exercício 2: Carrinho de Compras}
    \textbf{Contexto:} Sistema com descontos e cupons
    
    \vspace{1em}
    
    \textbf{Sua tarefa:}
    \begin{itemize}
        \item Escrever testes que cubram todos os cenários
        \item Analisar cobertura com coverage.py
        \item Identificar código não coberto
        \item Criar testes para atingir 100\% de cobertura
    \end{itemize}
    
    \vspace{1em}
    
    \begin{alertblock}{Desafio}
        Após rodar \texttt{coverage report}, identifique as linhas não cobertas e crie testes para elas!
    \end{alertblock}
\end{frame}

\begin{frame}[fragile]{Exercício 3: TDD - Calculadora de IMC}
    \textbf{Contexto:} Implementar seguindo Test-Driven Development
    
    \vspace{1em}
    
    \textbf{Ciclo TDD:}
    \begin{enumerate}
        \item \textcolor{red}{\textbf{RED}} - Escrever teste que falha
        \item \textcolor{green!70!black}{\textbf{GREEN}} - Implementar código mínimo para passar
        \item \textcolor{blue}{\textbf{REFACTOR}} - Melhorar o código
    \end{enumerate}
    
    \vspace{1em}
    
    \textbf{Funcionalidades:}
    \begin{itemize}
        \item Calcular IMC: peso / altura²
        \item Classificar: Abaixo do peso, Normal, Sobrepeso, Obesidade
        \item Validar entradas (peso e altura > 0)
    \end{itemize}
\end{frame}

%=============================================================================
\section{Integração com o Projeto}
%=============================================================================

\begin{frame}{Requisitos para U3}
    \begin{block}{Automação de Testes (30\%)}
        \begin{itemize}
            \item Mínimo 15 testes unitários
            \item Mínimo 5 testes de integração
            \item Seguir padrão AAA
            \item Princípios FIRST
        \end{itemize}
    \end{block}
    
    \begin{block}{Cobertura de Código (20\%)}
        \begin{itemize}
            \item Cobertura de linhas $\geq$ 70\%
            \item Cobertura de branches $\geq$ 60\%
            \item Módulos críticos $\geq$ 85\%
            \item Justificar código não coberto
        \end{itemize}
    \end{block}
\end{frame}

% \begin{frame}{Estrutura de Testes no Projeto}
%     \begin{block}{Organização Recomendada}
%         \begin{verbatim}
% projeto/
% ├── src/
% │   └── (código fonte)
% ├── tests/
% │   ├── unit/
% │   │   ├── test_modelo1.py
% │   │   └── test_servicos.py
% │   ├── integration/
% │   │   └── test_fluxo.py
% │   └── conftest.py
% └── README.md
%         \end{verbatim}
%     \end{block}
% \end{frame}

\begin{frame}{Comandos Essenciais}
    \begin{block}{pytest}
        \texttt{pytest -v} \hfill Executar com verbose \\
        \texttt{pytest tests/unit/} \hfill Testar pasta específica \\
        \texttt{pytest -k "senha"} \hfill Testar por nome \\
        \texttt{pytest --maxfail=1} \hfill Parar no primeiro erro
    \end{block}
    
    \begin{block}{coverage.py}
        \texttt{coverage run -m pytest} \hfill Executar com cobertura \\
        \texttt{coverage report} \hfill Relatório no terminal \\
        \texttt{coverage html} \hfill Relatório HTML detalhado \\
        \texttt{coverage report --show-missing} \hfill Mostrar linhas não cobertas
    \end{block}
\end{frame}

%=============================================================================
\section{Conclusão}
%=============================================================================

\begin{frame}{Principais Aprendizados}
    \begin{enumerate}
        \item Testes automatizados são \textbf{investimento}, não custo
        \item Siga a \textbf{pirâmide de testes}: 70\% unitários
        \item Use \textbf{FIRST} como guia de qualidade
        \item Cobertura é uma \textbf{métrica}, não uma meta absoluta
        \item \textbf{TDD} força design melhor e testes mais focados
    \end{enumerate}
    
    \vspace{1em}
    
    \begin{alertblock}{Lembre-se}
        ``Código sem testes é código legado por definição'' \\
        \hfill --- Michael Feathers
    \end{alertblock}
\end{frame}

\begin{frame}{Recursos e Referências}
    \textbf{Documentação:}
    \begin{itemize}
        \item pytest: \url{https://docs.pytest.org/}
        \item coverage.py: \url{https://coverage.readthedocs.io/}
        \item Jest (JS): \url{https://jestjs.io/}
    \end{itemize}
    
    \vspace{1em}
    
    \textbf{Livros:}
    \begin{itemize}
        \item Clean Code, Robert C. Martin (Cap. 9)
        \item Test Driven Development, Kent Beck
        \item The Art of Unit Testing, Roy Osherove
    \end{itemize}
\end{frame}

\begin{frame}{Próximos Passos}
    \begin{block}{Hoje}
        \begin{itemize}
            \item Completar os 3 exercícios práticos
            \item Experimentar com coverage.py
        \end{itemize}
    \end{block}
    
    \begin{block}{Esta Semana}
        \begin{itemize}
            \item Adicionar testes ao seu projeto
            \item Analisar cobertura atual
            \item Identificar gaps de teste
        \end{itemize}
    \end{block}
    
    \begin{block}{Para U3 (05/12)}
        \begin{itemize}
            \item Suite completa de testes
            \item Cobertura $\geq$ 70\%
            \item Documentação de qualidade
        \end{itemize}
    \end{block}
\end{frame}

\begin{frame}{Contato e Suporte}
    \begin{center}
        \Large
        \textbf{Dúvidas?}
        
        \vspace{2em}
        
        \faEnvelope\ Atendimento: Segundas 14h-16h (online) \\
        \vspace{0.5em}
        \faDiscord\ Discord: \url{https://discord.gg/bbMFJBQRT8} \\
        \vspace{0.5em}
        \faGithub\ GitHub: \url{https://github.com/fmarquesfilho/bpp-2025-2}
        
        \vspace{2em}
        
        \Huge
        Bons testes! \faRocket
    \end{center}
\end{frame}

\end{document}
