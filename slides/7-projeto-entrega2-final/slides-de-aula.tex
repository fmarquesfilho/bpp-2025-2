\documentclass[aspectratio=169]{beamer}
\usepackage[utf8]{inputenc}
\usepackage[T1]{fontenc}
\usepackage[brazilian]{babel}
\usepackage{listings}
\usepackage{xcolor}
\usepackage{tikz}
\usepackage{fontawesome5}
\usetikzlibrary{shapes,arrows,positioning}

% Tema
\usetheme{Madrid}
\usecolortheme{default}

% Cores personalizadas
\definecolor{codecolor}{rgb}{0.95,0.95,0.95}
\definecolor{darkgreen}{rgb}{0,0.5,0}

% Configuração de código
\lstset{
    basicstyle=\ttfamily\footnotesize,
    backgroundcolor=\color{codecolor},
    frame=single,
    breaklines=true,
    showstringspaces=false,
    tabsize=2
}

% Informações do documento
\title{Entrega Final do Projeto}
\subtitle{Unidade 3 - MVP Final com Testes e Análises}
\author{Boas Práticas de Programação - 2025.2}
\institute{UFRN}
\date{Entrega: 05/12/2025 às 23:59}

\begin{document}

% Slide de título
\frame{\titlepage}

%=============================================================================
\begin{frame}{U3: Evolução do Projeto}
    \begin{columns}
        \begin{column}{0.5\textwidth}
            \textbf{U1 focou em:}
            \begin{itemize}
                \item \faCheck\ Planejamento e MVP
                \item \faCheck\ Código Limpo
                \item \faCheck\ Code Smells
                \item \faCheck\ Refatoração
            \end{itemize}
        \end{column}
        \begin{column}{0.5\textwidth}
            \textbf{U3 adiciona:}
            \begin{itemize}
                \item \faPlus\ Automação de Testes
                \item \faPlus\ Análise de Cobertura
                \item \faPlus\ Técnicas de Depuração
                \item \faPlus\ Análise de Desempenho
                \item \faPlus\ Gerenciamento de Memória
            \end{itemize}
        \end{column}
    \end{columns}
    
    \vspace{1em}
    
    \begin{alertblock}{Importante}
        U3 é a continuação do mesmo projeto da U1! Você vai adicionar novas análises ao MVP existente.
    \end{alertblock}
\end{frame}

%=============================================================================
\begin{frame}{Distribuição da Nota - U3}
    \begin{center}
        \begin{tikzpicture}[scale=0.9]
            % Pie chart
            \def\radius{2.5}
            
            % Automação de Testes (30%)
            \fill[blue!40] (0,0) -- (0:\radius) arc (0:108:\radius) -- cycle;
            \node at (54:1.5) {\textbf{30\%}};
            \node[align=center] at (54:3.2) {\small Automação\\de Testes};
            
            % Cobertura (20%)
            \fill[green!40] (0,0) -- (108:\radius) arc (108:180:\radius) -- cycle;
            \node at (144:1.5) {\textbf{20\%}};
            \node[align=center] at (144:3.2) {\small Cobertura\\de Código};
            
            % Depuração (15%)
            \fill[yellow!60] (0,0) -- (180:\radius) arc (180:234:\radius) -- cycle;
            \node at (207:1.5) {\textbf{15\%}};
            \node[align=center] at (207:3.2) {\small Depuração};
            
            % Performance (20%)
            \fill[orange!50] (0,0) -- (234:\radius) arc (234:306:\radius) -- cycle;
            \node at (270:1.5) {\textbf{20\%}};
            \node[align=center] at (270:3.2) {\small Análise de\\Desempenho};
            
            % Memória (15%)
            \fill[red!40] (0,0) -- (306:\radius) arc (306:360:\radius) -- cycle;
            \node at (333:1.5) {\textbf{15\%}};
            \node[align=center] at (333:3.2) {\small Gerenciamento\\de Memória};
        \end{tikzpicture}
    \end{center}
\end{frame}

%=============================================================================
\begin{frame}{1. Automação de Testes (30\%)}
    \begin{block}{Requisitos Mínimos}
        \begin{itemize}
            \item \textbf{15+ testes unitários} cobrindo funcionalidades principais
            \item \textbf{5+ testes de integração} testando interação entre componentes
            \item Seguir padrão \textbf{AAA} (Arrange-Act-Assert)
            \item Princípios \textbf{FIRST} (Fast, Independent, Repeatable, Self-validating, Timely)
        \end{itemize}
    \end{block}
    
    \vspace{0.5em}
    
    \begin{alertblock}{Qualidade > Quantidade}
        Testes bem escritos e significativos valem mais que muitos testes triviais!
    \end{alertblock}
\end{frame}

%=============================================================================
\begin{frame}{2. Análise de Cobertura (20\%)}
    \begin{block}{Métricas Obrigatórias}
        \begin{itemize}
            \item \textbf{Cobertura de Linhas:} Mínimo 70\%
            \item \textbf{Cobertura de Branches:} Mínimo 60\%
            \item \textbf{Módulos Críticos:} Mínimo 85\%
        \end{itemize}
    \end{block}
    
    \vspace{0.5em}
    
    \begin{block}{Entregáveis}
        \begin{enumerate}
            \item Relatório HTML de cobertura na pasta \texttt{tests/coverage-results/}
            \item Screenshot das métricas principais
            \item Documento \texttt{coverage-report.md} com:
            \begin{itemize}
                \item Evolução da cobertura (início vs final)
                \item Justificativa para código não coberto (se < 80\%)
            \end{itemize}
        \end{enumerate}
    \end{block}
\end{frame}

%=============================================================================
\begin{frame}[fragile]{3. Técnicas de Depuração (15\%)}
    \begin{block}{Requisito: Documentar 3+ bugs}
        Para cada bug, documente em \texttt{docs/debugging-log.md}:
    \end{block}
    
    \begin{enumerate}
        \item \textbf{Identificação}: Data, módulo, severidade
        \item \textbf{Descrição}: Como reproduzir o problema
        \item \textbf{Investigação}: Técnica de depuração usada
        \item \textbf{Causa raiz}: O que estava errado
        \item \textbf{Correção}: Código antes/depois
        \item \textbf{Verificação}: Testes adicionados
    \end{enumerate}
    
    \vspace{0.5em}
    
    \begin{exampleblock}{Técnicas esperadas (escolha livre)}
        Debugger, logging, testes automatizados, binary search, stack trace, git bisect
    \end{exampleblock}
\end{frame}

%=============================================================================
\begin{frame}{4. Análise de Desempenho (20\%)}
    \begin{block}{Requisito: Identificar 2+ gargalos}
        Para cada gargalo, documente em \texttt{docs/performance-analysis.md}:
    \end{block}
    
    \begin{columns}
        \begin{column}{0.5\textwidth}
            \textbf{Análise:}
            \begin{itemize}
                \item Módulo/função afetada
                \item Problema identificado
                \item Medição inicial (tempo)
                \item Complexidade (Big O)
            \end{itemize}
        \end{column}
        \begin{column}{0.5\textwidth}
            \textbf{Otimização:}
            \begin{itemize}
                \item Técnica aplicada
                \item Código antes/depois
                \item Medição final
                \item Ganho de performance
                \item Trade-offs
            \end{itemize}
        \end{column}
    \end{columns}
    
    \vspace{0.5em}
    
    \begin{alertblock}{Ferramentas}
        Python: cProfile, timeit | JS: Chrome DevTools | Java: JProfiler | C/C++: gprof, Valgrind
    \end{alertblock}
\end{frame}

%=============================================================================
\begin{frame}{5. Gerenciamento de Memória (15\%)}
    \begin{columns}
        \begin{column}{0.5\textwidth}
            \textbf{C/C++ (obrigatório):}
            \begin{itemize}
                \item Usar Valgrind
                \item Detectar memory leaks
                \item Corrigir acessos inválidos
                \item Documentar correções
            \end{itemize}
        \end{column}
        \begin{column}{0.5\textwidth}
            \textbf{Outras linguagens:}
            \begin{itemize}
                \item Uso eficiente de estruturas
                \item Evitar referências circulares
                \item Generators para grandes volumes
                \item Cache com limite (LRU)
            \end{itemize}
        \end{column}
    \end{columns}
    
    \vspace{1em}
    
    \begin{block}{Documento: \texttt{docs/memory-analysis.md}}
        \begin{itemize}
            \item Análise realizada (ferramenta usada)
            \item Problemas encontrados
            \item Otimizações implementadas
            \item Medições antes/depois
        \end{itemize}
    \end{block}
\end{frame}

%=============================================================================
\begin{frame}[fragile]{Estrutura de Entrega}
    \begin{lstlisting}
entrega-u3/
|-- README.md                  # Visao geral + instrucoes
|-- docs/
|   |-- testing-report.md      # Relatorio completo
|   |-- coverage-report.md     # Analise de cobertura
|   |-- debugging-log.md       # Bugs encontrados
|   |-- performance-analysis.md # Gargalos e otimizacoes
|   `-- memory-analysis.md     # Analise de memoria
|-- tests/
|   |-- unit/                  # Testes unitarios
|   |-- integration/           # Testes de integracao
|   `-- coverage-results/      # Relatorios HTML
|-- src/
|   `-- (codigo fonte refatorado)
`-- video-presentation.md      # Link para video
    \end{lstlisting}
\end{frame}

%=============================================================================
\begin{frame}{Vídeo de Apresentação (10-12 min)}
    \begin{enumerate}
        \item \textbf{Recapitulação} (1-2 min)
        \begin{itemize}
            \item Visão do produto
            \item Evolução desde U1
        \end{itemize}
        
        \item \textbf{Demonstração de Testes} (3-4 min)
        \begin{itemize}
            \item Execução da suite
            \item Relatório de cobertura
        \end{itemize}
        
        \item \textbf{Depuração} (2 min)
        \begin{itemize}
            \item 1-2 bugs interessantes
            \item Como foram corrigidos
        \end{itemize}
        
        \item \textbf{Performance} (2 min)
        \begin{itemize}
            \item Gargalos encontrados
            \item Métricas before/after
        \end{itemize}
        
        \item \textbf{Conclusão} (2-3 min)
        \begin{itemize}
            \item Qualidade final
            \item Lições aprendidas
        \end{itemize}
    \end{enumerate}
\end{frame}

%=============================================================================
\begin{frame}{Checklist de Entrega}
    \begin{columns}
        \begin{column}{0.5\textwidth}
            \textbf{Documentação:}
            \begin{itemize}
                \item[$\square$] README.md atualizado
                \item[$\square$] testing-report.md
                \item[$\square$] coverage-report.md
                \item[$\square$] debugging-log.md (3+ bugs)
                \item[$\square$] performance-analysis.md (2+ gargalos)
                \item[$\square$] memory-analysis.md
            \end{itemize}
        \end{column}
        \begin{column}{0.5\textwidth}
            \textbf{Código e Testes:}
            \begin{itemize}
                \item[$\square$] 10+ testes unitários
                \item[$\square$] Todos passando
                \item[$\square$] Cobertura $\geq$ 70\%
                \item[$\square$] Relatório HTML incluído
            \end{itemize}
            
            \vspace{0.5em}
            
            \textbf{Apresentação:}
            \begin{itemize}
                \item[$\square$] Vídeo 10-12 min
                \item[$\square$] Link funcionando
            \end{itemize}
        \end{column}
    \end{columns}
\end{frame}

%=============================================================================
\begin{frame}{Informações Finais}
    \begin{block}{Entrega}
        \textbf{Data:} 05/12/2025 às 23:59 \\
        \textbf{Plataforma:} SIGAA - Tarefa "Entrega U3" \\
        \textbf{Formato:} ZIP único: \texttt{u3-[projeto]-[integrantes].zip}
    \end{block}
    
    \begin{block}{Trabalho em Grupo}
        \begin{itemize}
            \item Individual ou grupos de 2-4 pessoas
            \item Descrever contribuição de cada membro
            \item Todos devem dominar o código completo
        \end{itemize}
    \end{block}
    
    \begin{alertblock}{Plantão de Dúvidas}
        \textbf{05/12 no horário da aula} (online) - Estarei disponível para tirar dúvidas sobre a entrega
    \end{alertblock}
\end{frame}

%=============================================================================
\begin{frame}{Suporte e Recursos}
    \begin{center}
        \Large
        \textbf{Precisa de ajuda?}
        
        \vspace{2em}
        
        \normalsize
        \faEnvelope\ \textbf{Atendimento:} Segundas 14h-16h (online) \\
        \vspace{0.5em}
        \faDiscord\ \textbf{Discord:} \url{https://discord.gg/bbMFJBQRT8} \\
        \vspace{0.5em}
        \faGithub\ \textbf{GitHub:} \url{https://github.com/fmarquesfilho/bpp-2025-2}
        
        \vspace{3em}
        
        \Huge
        Bom trabalho!
    \end{center}
\end{frame}

\end{document}
