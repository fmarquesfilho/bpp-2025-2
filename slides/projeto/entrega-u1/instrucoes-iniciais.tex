\documentclass[10pt]{beamer}
\usepackage[utf8]{inputenc}
\usepackage[T1]{fontenc}
\usepackage[brazilian]{babel}
\usepackage{tikz}
\usetikzlibrary{shapes.geometric, arrows, positioning}
\usepackage{amssymb}
\usetheme{Madrid}

\title{Boas Práticas de Programação (2025.2)}
\subtitle{Orientações para a Primeira Entrega - Unidade 1}
\author{Prof. Fernando Figueira}
\institute{DIMAp - UFRN}
\date{Setembro de 2025}

\begin{document}

\frame{\titlepage}

\begin{frame}{Objetivo da Unidade 1}
\begin{itemize}
    \item Definir a visão do produto.
    \item Estruturar o backlog priorizado.
    \item Escolher a stack tecnológica.
    \item Preparar a entrega via SIGAA.
\end{itemize}
\end{frame}

\begin{frame}{Elementos da Visão do Produto}
\begin{itemize}
    \item Propósito e objetivos.
    \item Usuários-alvo.
    \item Valor proposto.
    \item Diferenciação.
    \item Escopo do MVP.
\end{itemize}
\end{frame}

\begin{frame}{Exemplo de Visão do Produto}
\centering
\begin{tikzpicture}
    \node[draw, fill=blue!10, rounded corners, text width=8cm, align=left] {
        \textbf{Sistema de Gestão de Tarefas Pessoais} \\
        - Propósito: Ajudar usuários a organizarem tarefas do dia a dia. \\
        - Usuários: Pessoas que buscam produtividade. \\
        - Valor: Simplicidade e eficiência. \\
        - Diferenciação: Interface limpa e notificações inteligentes. \\
        - MVP: Criar, listar e marcar tarefas como concluídas.
    };
\end{tikzpicture}
\end{frame}

\begin{frame}{Backlog Priorizado}
\centering
\begin{tabular}{|c|l|}
\hline
\textbf{Prioridade} & \textbf{Funcionalidade} \\
\hline
Alta & Cadastrar tarefa \\
Alta & Listar tarefas \\
Média & Marcar tarefa como concluída \\
Baixa & Editar tarefa \\
\hline
\end{tabular}
\end{frame}

\begin{frame}{Preparação para Entrega}
\begin{itemize}
    \item Documentos em PDF.
    \item Vídeo de 5–8 minutos.
    \item Nomes de arquivos padronizados.
    \item Entrega via SIGAA até 02/10/2025.
\end{itemize}
\end{frame}

\begin{frame}{Próximos Passos}
\begin{itemize}
    \item Escolher a linguagem e ferramentas.
    \item Configurar repositório Git.
    \item Iniciar desenvolvimento incremental.
    \item Buscar ajuda nas aulas de dúvidas.
\end{itemize}
\end{frame}

\end{document}
