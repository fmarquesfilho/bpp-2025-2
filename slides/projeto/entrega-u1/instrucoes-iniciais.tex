\documentclass[10pt]{beamer}
\usepackage[utf8]{inputenc}
\usepackage[T1]{fontenc}
\usepackage[brazilian]{babel}
\usepackage{tikz}
\usetikzlibrary{shapes.geometric, arrows, positioning, matrix}
\usepackage{amssymb}
\usepackage{xcolor}
\usetheme{Madrid}
\usepackage{pifont}

\title{Boas Práticas de Programação (2025.2)}
\subtitle{Planejamento Estratégico de Projeto com Princípios Ágeis}
\author{Prof. Fernando Figueira}
\institute{DIMAp - UFRN}
\date{Setembro de 2025}

% Definindo estilos personalizados
\tikzset{
    process/.style = {rectangle, rounded corners, minimum width=3cm, minimum height=1cm, text centered, draw=black, fill=blue!20},
    priority1/.style = {rectangle, rounded corners, minimum width=8cm, minimum height=0.8cm, text centered, draw=black, fill=green!30},
    priority2/.style = {rectangle, rounded corners, minimum width=8cm, minimum height=0.8cm, text centered, draw=black, fill=yellow!30},
    priority3/.style = {rectangle, rounded corners, minimum width=8cm, minimum height=0.8cm, text centered, draw=black, fill=orange!30},
    arrow/.style = {thick,->,>=stealth}
}

\begin{document}

\frame{\titlepage}

% Slide 1 - Overview do Planejamento
\begin{frame}{Por que Planejamento Estratégico?}
\begin{columns}[c]
\column{0.5\textwidth}
\begin{itemize}
    \item \textbf{Foco no Valor}: Resolver problemas reais
    \item \textbf{Gestão de Escopo}: Evitar feature creep
    \item \textbf{Entrega Incremental}: Feedback constante
    \item \textbf{Aprendizado Ágil}: Adaptar baseado em evidências
\end{itemize}

\column{0.5\textwidth}
\centering
\begin{tikzpicture}[scale=0.7]
    \draw[fill=blue!20] (0,0) circle (2cm) node {\textbf{Seu Projeto}};
    \draw[fill=green!20] (-1.5,1.5) circle (0.8cm) node[font=\tiny] {Valor};
    \draw[fill=yellow!20] (1.5,1.5) circle (0.8cm) node[font=\tiny] {Escopo};
    \draw[fill=orange!20] (1.5,-1.5) circle (0.8cm) node[font=\tiny] {Tempo};
    \draw[fill=red!20] (-1.5,-1.5) circle (0.8cm) node[font=\tiny] {Qualidade};
\end{tikzpicture}
\end{columns}
\end{frame}

% Slide 2 - Visão do Produto com Template
\begin{frame}{Visão do Produto: Template Prático}
\begin{block}{Template da Visão}
\textbf{Para} [usuários-alvo] \\
\textbf{Que} [problema/necessidade] \\
\textbf{O} [nome do produto] \textbf{é um} [categoria] \\
\textbf{Que} [benefício principal] \\
\textbf{Diferente de} [alternativa existente] \\
\textbf{Nosso produto} [diferencial único]
\end{block}

\vspace{0.3cm}
\begin{exampleblock}{Checklist da Visão}
\small
\checkmark\ Define usuário-alvo específico \\
\checkmark\ Identifica problema concreto \\
\checkmark\ Explicita valor único \\
\checkmark\ É inspiradora mas realista \\
\checkmark\ Cabe no escopo acadêmico
\end{exampleblock}
\end{frame}

% Slide 3 - Exemplo de Visão
\begin{frame}{Exemplo: Visão do Produto}
\begin{alertblock}{Sistema de Controle de Gastos Universitário}
\textbf{Para} estudantes universitários \\
\textbf{Que} têm dificuldade em controlar gastos mensais \\
\textbf{O} UniBudget \textbf{é uma} aplicação de controle financeiro \\
\textbf{Que} permite registro rápido e visualização de padrões \\
\textbf{Diferente de} apps complexos como Mobills \\
\textbf{Nosso produto} foca na simplicidade e contexto estudantil
\end{alertblock}

\vspace{0.3cm}
\begin{itemize}
    \item \textbf{Usuário}: Específico e bem definido
    \item \textbf{Problema}: Concreto e relevante
    \item \textbf{Diferencial}: Simplicidade + contexto
\end{itemize}
\end{frame}



% Slide 5 - Framework de MVP
\begin{frame}{Framework para Definir MVP}
\begin{block}{1. Problema Core}
Qual o \textbf{principal} problema que seu produto resolve?
\end{block}

\begin{block}{2. Hipótese de Valor}
"Acreditamos que [usuários] vão [comportamento] porque [benefício]"
\end{block}

\begin{block}{3. Métricas de Sucesso}
Como você saberá se funcionou?
\end{block}

\vspace{0.3cm}
\begin{exampleblock}{Exemplo - Sistema de Biblioteca}
\textbf{Problema}: Estudantes perdem tempo procurando livros \\
\textbf{Hipótese}: Vão consultar antes de ir à biblioteca \\
\textbf{Métrica}: Redução de idas "em vão" à biblioteca
\end{exampleblock}
\end{frame}

% Slide 6 - Técnica MoSCoW
\begin{frame}{Técnica MoSCoW para Definir Escopo}
\centering
\begin{tikzpicture}[scale=0.9]
    \node[priority1, text width=7cm] at (0,0) {\textbf{Must Have} - Essencial para o MVP};
    \node[priority2, text width=7cm] at (0,-1) {\textbf{Should Have} - Importante, versão 2.0};
    \node[priority3, text width=7cm] at (0,-2) {\textbf{Could Have} - Desejável, backlog futuro};
    \node[draw, fill=red!20, rounded corners, minimum width=7cm, minimum height=0.8cm, text centered] at (0,-3) {\textbf{Won't Have} - Explicitamente excluído};
\end{tikzpicture}

\vspace{0.5cm}
\textbf{Dica}: Se você tem dúvidas se algo é Must/Should, provavelmente é Should!
\end{frame}

% Slide 7 - Exemplo MoSCoW Aplicado
\begin{frame}{Exemplo: MoSCoW - Sistema de Biblioteca}
\begin{columns}[t]
\column{0.48\textwidth}
\textbf{Must Have (MVP)}
\begin{itemize}
    \item Cadastrar livro
    \item Listar livros
    \item Busca simples
    \item Marcar emprestado/disponível
\end{itemize}

\textbf{Should Have (v2.0)}
\begin{itemize}
    \item Filtros avançados
    \item Histórico de empréstimos
    \item Dados do usuário
\end{itemize}

\column{0.48\textwidth}
\textbf{Could Have}
\begin{itemize}
    \item Sistema de reservas
    \item Notificações
    \item Estatísticas
    \item API externa
\end{itemize}

\textbf{Won't Have}
\begin{itemize}
    \item Pagamentos de multa
    \item Integração com biblioteca física
    \item App mobile
\end{itemize}
\end{columns}
\end{frame}

% Slide 8 - Backlog Estruturado
\begin{frame}{Product Backlog: Organização Prática}
\textbf{Estrutura de cada item:}
\begin{center}
[Prioridade] + [User Story] + [Critérios] + [Estimativa]
\end{center}

\vspace{0.3cm}
\textbf{User Story Template:}
\begin{alertblock}{}
Como [tipo de usuário], quero [funcionalidade] para [benefício]\\
Exemplo: "Como estudante, quero buscar livros por autor para encontrar obras específicas rapidamente"
\end{alertblock}

\vspace{0.3cm}
\textbf{Critérios de Aceitação:}
\begin{itemize}
    \item Condições que a funcionalidade deve atender
    \item Cenários de teste implícitos
    \item Definição clara de "pronto"
\end{itemize}
\end{frame}

% Slide 9 - Exemplo Completo de Backlog
\begin{frame}{Exemplo: Backlog Completo}
\scriptsize
\begin{tabular}{|c|p{5cm}|p{3cm}|c|}
\hline
\textbf{Pri} & \textbf{User Story} & \textbf{Critérios de Aceitação} & \textbf{Est} \\
\hline
P1 & Como estudante, quero cadastrar uma nova tarefa para não esquecer & - Campos: título, descrição, data\newline - Validação obrigatórios\newline - Mensagem de confirmação & 4h \\
\hline
P1 & Como estudante, quero ver todas as tarefas para ter visão geral & - Lista ordenada por data\newline - Indicador de urgência\newline - Máximo 50 por tela & 3h \\
\hline
P2 & Como estudante, quero filtrar por status para focar no importante & - Filtros: todas, pendentes, feitas\newline - Filtro persiste na sessão & 2h \\
\hline
\end{tabular}

\vspace{0.2cm}
\textbf{Observe}: User stories focam no valor, não na implementação!
\end{frame}

% Slide 10 - Técnicas de Priorização
\begin{frame}{Matriz Valor x Esforço para Priorização}
\centering
\begin{tikzpicture}[scale=1.1]
    % Eixos
    \draw[thick,->] (0,0) -- (5,0) node[right] {Esforço};
    \draw[thick,->] (0,0) -- (0,4) node[above] {Valor};
    
    % Quadrantes
    \fill[green!20] (0,2) rectangle (2.5,4);
    \fill[yellow!20] (2.5,2) rectangle (5,4);
    \fill[blue!20] (0,0) rectangle (2.5,2);
    \fill[red!20] (2.5,0) rectangle (5,2);
    
    % Labels dos quadrantes
    \node[text width=2cm, align=center] at (1.25,3) {\textbf{FAÇA\\PRIMEIRO}\\\small P1};
    \node[text width=2cm, align=center] at (3.75,3) {\textbf{PLANEJE\\BEM}\\\small P2};
    \node[text width=2cm, align=center] at (1.25,1) {\textbf{SE SOBRAR\\TEMPO}\\\small P3};
    \node[text width=2cm, align=center] at (3.75,1) {\textbf{NÃO FAÇA\\(agora)}\\\small Won't};
    
    % Exemplos
    \node[fill=white, circle, draw] at (1,3.5) {\tiny Login};
    \node[fill=white, circle, draw] at (4,3.5) {\tiny API};
    \node[fill=white, circle, draw] at (1,0.5) {\tiny Log};
    \node[fill=white, circle, draw] at (4,0.5) {\tiny IA};
\end{tikzpicture}
\end{frame}


% Slide 12 - Cronograma Detalhado
\begin{frame}{Cronograma: Setembro 2025 (Unidade 1)}
\scriptsize
\begin{tikzpicture}[scale=0.85]
    % Semanas
    \draw[fill=blue!10] (0,3) rectangle (2.5,4) node[midway, font=\tiny] {Sem 1: Setup};
    \draw[fill=green!20] (2.7,3) rectangle (5.2,4) node[midway, font=\tiny] {Sem 2: Sprint 1};
    \draw[fill=green!20] (5.4,3) rectangle (7.9,4) node[midway, font=\tiny] {Sem 3: Sprint 2};
    \draw[fill=yellow!20] (8.1,3) rectangle (10.6,4) node[midway, font=\tiny] {Sem 4: Sprint 3};
    \draw[fill=red!20] (10.8,3) rectangle (13.3,4) node[midway, font=\tiny] {Sem 5: Entrega};
    
    % Atividades detalhadas
    \node[below, font=\tiny, text width=2.3cm, align=center] at (1.25,3) {
        • Visão do produto\\
        • Backlog inicial\\
        • Setup ambiente
    };
    \node[below, font=\tiny, text width=2.3cm, align=center] at (3.95,3) {
        • Funcionalidades P1\\
        • Primeira versão\\
        • Testes básicos
    };
    \node[below, font=\tiny, text width=2.3cm, align=center] at (6.65,3) {
        • Completar P1\\
        • Tratamento erros\\
        • Melhorar UX
    };
    \node[below, font=\tiny, text width=2.3cm, align=center] at (9.35,3) {
        • Funcionalidades P2\\
        • Refatoração\\
        • Documentação
    };
    \node[below, font=\tiny, text width=2.3cm, align=center] at (12.05,3) {
        • Vídeo\\
        • Revisão docs\\
        • Upload SIGAA
    };
\end{tikzpicture}
\normalsize

\vspace{0.3cm}
\textbf{Entrega}: 02/10/2025 até 23:59
\end{frame}

% Slide 13 - Definition of Done
\begin{frame}{Definition of Done (DoD) Progressivo}
\begin{columns}[t]
\column{0.32\textwidth}
\textbf{Sprint 1}
\begin{itemize}
    \item Funcionalidade implementada
    \item Código compila
    \item Teste manual OK
    \item Commit descritivo
\end{itemize}

\column{0.32\textwidth}
\textbf{Sprint 2}
\begin{itemize}
    \item Tudo do Sprint 1 +
    \item Tratamento de erros
    \item Código comentado
    \item Refatoração inicial
\end{itemize}

\column{0.32\textwidth}
\textbf{Sprint 3}
\begin{itemize}
    \item Tudo do Sprint 2 +
    \item Testes automatizados
    \item Documentação completa
    \item Code review próprio
\end{itemize}
\end{columns}

\vspace{0.5cm}
\begin{alertblock}{Dica}
DoD evolui ao longo do projeto. Comece simples e melhore gradualmente!
\end{alertblock}
\end{frame}

% Slide 14 - Ferramentas Práticas
\begin{frame}{Ferramentas para Gestão do Projeto}
\begin{columns}[c]
\column{0.5\textwidth}
\textbf{Gestão de Backlog:}
\begin{itemize}
    \item \textbf{Trello}: Visual e simples
    \item \textbf{GitHub Projects}: Integrado com código
    \item \textbf{Notion}: Robusto para docs
    \item \textbf{Google Sheets}: Rápido de usar
\end{itemize}

\textbf{Controle de Versão:}
\begin{itemize}
    \item Commits frequentes
    \item Mensagens descritivas
    \item Branches por feature
\end{itemize}

\column{0.5\textwidth}
\textbf{Estrutura de Commits:}
\begin{block}{Exemplo de commit}
\texttt{git commit -m "[TIPO] Descrição"}

\vspace{0.2cm}
\textbf{Tipos:}
\begin{itemize}
    \item \texttt{FEAT}: nova funcionalidade
    \item \texttt{FIX}: correção de bug
    \item \texttt{DOCS}: documentação
    \item \texttt{REFACTOR}: refatoração
    \item \texttt{TEST}: testes
\end{itemize}
\end{block}
\end{columns}
\end{frame}

% Slide 15 - Estrutura de Projeto
\begin{frame}{Estrutura de Pastas Recomendada}
\begin{columns}[c]
\column{0.5\textwidth}
\begin{block}{Estrutura Básica}
\texttt{projeto/}\\
\texttt{|-- src/} \hfill \textit{Código fonte}\\
\texttt{|-- tests/} \hfill \textit{Testes}\\
\texttt{|-- docs/} \hfill \textit{Documentação}\\
\texttt{|-- README.md} \hfill \textit{Visão geral}\\
\texttt{|-- CHANGELOG.md} \hfill \textit{Mudanças}
\end{block}

\column{0.5\textwidth}
\textbf{README.md essencial:}
\begin{itemize}
    \item Título do projeto
    \item Descrição breve
    \item Como instalar/executar
    \item Como usar
    \item Estrutura do código
    \item Autor e contato
\end{itemize}
\end{columns}

\vspace{0.3cm}
\begin{alertblock}{Dica}
README é o cartão de visitas do seu projeto. Invista tempo nele!
\end{alertblock}
\end{frame}



% Slide 17 - Sinais de Bom Projeto
\begin{frame}{Sinais de um Bom Projeto}
\begin{columns}[c]
\column{0.5\textwidth}
\textbf{Sinais Positivos:}
\begin{itemize}
    \item \checkmark\ Explica o valor em 30s
    \item \checkmark\ MVP pode ser usado por alguém real
    \item \checkmark\ Você está animado para desenvolvê-lo
    \item \checkmark\ Escopo é realista para o tempo
\end{itemize}

\column{0.5\textwidth}
\textbf{Sinais de atenção:}
\begin{itemize}
    \item \ding{55}\ MVP muito complexo
    \item \ding{55}\ Foco na tecnologia, não no usuário
    \item \ding{55}\ Backlog com 50+ itens
    \item \ding{55}\ Problema vago ou inexistente
\end{itemize}
\end{columns}

\vspace{0.5cm}
\begin{alertblock}{Lembre-se}
Melhor um projeto simples bem executado que um complexo pela metade!
\end{alertblock}
\end{frame}

% Slide 18 - Erros Comuns
\begin{frame}{Erros Comuns a Evitar}
\begin{enumerate}
    \item \textbf{MVP muito ambicioso}
    \begin{itemize}
        \item[\ding{55}] "Sistema completo de e-commerce"
        \item[\checkmark] "Catálogo simples com carrinho básico"
    \end{itemize}
    
    \item \textbf{Foco na tecnologia}
    \begin{itemize}
        \item[\ding{55}] "Quero usar React, Node, MongoDB..."
        \item[\checkmark] "Como resolver o problema do usuário?"
    \end{itemize}
    
    \item \textbf{Backlog estático}
    \begin{itemize}
        \item[\ding{55}] "Planejei tudo, não vou mudar"
        \item[\checkmark] "Vou adaptar conforme aprendo"
    \end{itemize}
\end{enumerate}

\vspace{0.3cm}
\textbf{Mantra}: Ação e feedback são mais valiosos que planejamento excessivo
\end{frame}

% Slide 19 - Recursos de Apoio
\begin{frame}{Recursos de Apoio Disponíveis}
\textbf{Durante o Desenvolvimento:}
\begin{itemize}
    \item \textbf{Atendimentos}: Segundas 14h-16h (online) é bom?
    \item \textbf{Fórum da Disciplina}: Dúvidas técnicas e conceituais
    \item \textbf{GitHub do Curso}: Templates e exemplos
    \item \textbf{Material Complementar}: Artigos sobre MVP e backlog
\end{itemize}

\vspace{0.3cm}
\textbf{Templates Disponíveis:}
\begin{itemize}
    \item Documento de Visão do Produto
    \item Planilha de Product Backlog
    \item Estrutura de README.md
    \item Checklist de Definition of Done
\end{itemize}

\vspace{0.3cm}
\begin{alertblock}{Importante}
Não hesite em buscar ajuda! Melhor esclarecer dúvidas cedo que descobrir problemas na entrega.
\end{alertblock}
\end{frame}

% Slide 20 - Preparação da Entrega
\begin{frame}{Preparação da Entrega Final}
\textbf{Documentos Obrigatórios:}
\begin{itemize}
    \item Visão do Produto (PDF, 2-3 páginas)
    \item Product Backlog (PDF ou planilha)
    \item Vídeo de apresentação (5-8 minutos)
\end{itemize}

\vspace{0.3cm}
\textbf{Estrutura do Vídeo:}
\begin{itemize}
    \item \textbf{Min 1-2}: Problema e visão do produto
    \item \textbf{Min 3-4}: Demonstração do MVP planejado
    \item \textbf{Min 5-6}: Backlog e priorização
    \item \textbf{Min 7-8}: Stack tecnológica e próximos passos
\end{itemize}

\vspace{0.3cm}
\textbf{Formato dos Arquivos:}
\texttt{[TipoDoc]\_[NomeProjeto]\_[NomeAluno].pdf}
\end{frame}

% Slide 21 - Checklist Final
\begin{frame}{Checklist Final da Entrega}
\begin{columns}[t]
\column{0.5\textwidth}
\textbf{Antes de Enviar:}
\begin{itemize}
    \item Todos os arquivos no formato correto
    \item Links de vídeo acessíveis
    \item Documentos autoexplicativos
    \item Nomes seguem o padrão
    \item ZIP dentro do limite de tamanho
\end{itemize}

\column{0.5\textwidth}
\textbf{Autovaliação:}
\begin{itemize}
    \item Visão é clara e inspiradora
    \item MVP é realista e valioso
    \item Backlog está bem priorizado
    \item Vídeo comunica bem a ideia
    \item Estou orgulhoso do resultado
\end{itemize}
\end{columns}

\vspace{0.5cm}
\begin{block}{Entrega}
\textbf{SIGAA - Tarefa "Entrega U1"}\\
\textbf{Deadline}: 02/10/2025 até 23:59\\
\textbf{Formato}: ZIP único com todos os arquivos
\end{block}
\end{frame}

% Slide 22 - Próximos Passos
\begin{frame}{Próximos Passos}
\textbf{Esta Semana (02-08/09):}
\begin{itemize}
    \item Definir o problema que quer resolver
    \item Aplicar o template de visão do produto
    \item Criar backlog inicial com 10-15 itens
    \item Configurar ambiente de desenvolvimento
\end{itemize}

\vspace{0.3cm}
\textbf{Próxima Semana (09-15/09):}
\begin{itemize}
    \item Iniciar Sprint 1 com funcionalidades P1
    \item Setup do repositório Git
    \item Primeira versão funcional (mesmo simples)
    \item Buscar feedback de colegas
\end{itemize}

\vspace{0.3cm}
\textbf{Lembre-se}: O objetivo é aprender aplicando conceitos ágeis, não criar o projeto perfeito!
\end{frame}

% Slide Final
\begin{frame}{Valeu e Bom Trabalho!}
\centering
\begin{tikzpicture}
    \node[draw, fill=blue!10, rounded corners, minimum width=10cm, minimum height=4cm, text width=9cm, align=center] {
        \Large\textbf{O projeto é uma oportunidade de:}\\
        
        \vspace{0.5cm}
        
        Aplicar conceitos do curso na prática\\
        Aprender com feedback e iteração\\
        Desenvolver algo que você(s) tem orgulho\\
        
        \vspace{0.5cm}
    };
\end{tikzpicture}

\vspace{0.5cm}
\textbf{Dúvidas?} fernando@dimap.ufrn.br
\end{frame}

\end{document}
