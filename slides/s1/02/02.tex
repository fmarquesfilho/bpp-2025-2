\documentclass[10pt, aspectratio=169]{beamer}
\usepackage[utf8]{inputenc}
\usepackage[T1]{fontenc}
\usepackage[portuguese]{babel}
\usepackage{graphicx}
\usepackage{ragged2e}
\usepackage{hyperref}
\usepackage{booktabs}
\usepackage{tabularx}
\usepackage{colortbl}
\usepackage{xcolor}
\usepackage{tikz}
\usepackage{amssymb}
\usepackage{textcomp}
\usetikzlibrary{shapes,arrows,positioning}

\usetheme{Madrid}
\usecolortheme{seahorse}

\title{Por que estudar Boas Práticas de Programação?}
\subtitle{Introdução e Motivação}
\author{Prof. Fernando Figueira}
\institute{DIMAp -- UFRN}
\date{22 de Agosto de 2025}

\setbeamertemplate{footline}[frame number]
\setbeamertemplate{navigation symbols}{}

\definecolor{azulUni}{RGB}{0, 51, 102}
\setbeamercolor{palette primary}{bg=azulUni,fg=white}
\setbeamercolor{palette secondary}{bg=azulUni!80!black,fg=white}
\setbeamercolor{palette tertiary}{bg=azulUni!60!black,fg=white}
\setbeamercolor{palette quaternary}{bg=azulUni!40!black,fg=white}
\setbeamercolor{structure}{fg=azulUni}
\setbeamercolor{section in toc}{fg=azulUni}
\setbeamercolor{subsection in toc}{fg=azulUni!80!black}

\begin{document}

\begin{frame}
\titlepage
\end{frame}

\begin{frame}{Sumário}
\tableofcontents
\end{frame}

\section{Introdução}
\begin{frame}{O Dilema do Desenvolvedor}
\begin{columns}
\begin{column}{0.5\textwidth}
\begin{block}{Código que funciona $\neq$ Código bom}
\begin{itemize}
    \item Qualquer um pode escrever código que um computador entenda
    \item Bons programadores escrevem código que humanos entendam
    \item Manutenção consome 60-80\% do custo total de software
\end{itemize}
\end{block}
\end{column}
\begin{column}{0.5\textwidth}
\centering
% Ilustração de código espaguete feita com TikZ
\begin{tikzpicture}[scale=0.8]
  % Blocos de código
  \foreach \x/\y in {0.5/3, 1.5/2.8, 2.5/3.2, 3.5/2.9, 0.8/2.2, 2.2/2.1, 3.2/2.3, 1.2/1.4, 2.8/1.5}
    \fill[blue!60] (\x,\y) rectangle ++(0.3,0.2);
  
  % Conexões emaranhadas (código espaguete)
  \draw[red, thick, bend left=30] (0.5,3) to (2.2,2.1);
  \draw[red, thick, bend right=40] (1.5,2.8) to (3.2,2.3);
  \draw[red, thick, bend left=20] (2.5,3.2) to (1.2,1.4);
  \draw[red, thick, bend right=50] (3.5,2.9) to (0.8,2.2);
  \draw[red, thick, bend left=60] (0.8,2.2) to (2.8,1.5);
  \draw[red, thick, bend right=30] (2.2,2.1) to (3.5,2.9);
  
  % Símbolos de alerta
  \node[red, scale=1.5] at (1.5,0.8) {!};
  \node[red, scale=1.5] at (2.8,0.8) {!};
  
  \node[below] at (2,0.3) {\scriptsize{Código Espaguete}};
\end{tikzpicture}
\\\tiny{Fonte: Ilustração própria - Representação de código espaguete}
\end{column}
\end{columns}
\end{frame}

\begin{frame}{O que são Boas Práticas de Programação?}
\begin{block}{Definição}
Conjunto de técnicas, princípios e padrões que melhoram a qualidade, legibilidade e manutenibilidade do código, facilitando a colaboração e reduzindo erros.
\end{block}

\begin{columns}
\begin{column}{0.5\textwidth}
\begin{itemize}
    \item Nomenclatura clara
    \item Organização lógica
    \item Comentários úteis
    \item Formatação consistente
\end{itemize}
\end{column}
\begin{column}{0.5\textwidth}
\begin{itemize}
    \item Tratamento de erros
    \item Princípios de design
    \item Testes automatizados
    \item Refatoração contínua
\end{itemize}
\end{column}
\end{columns}
\end{frame}

\section{Impacto no Mundo Real}
\begin{frame}{Custo do Código de Baixa Qualidade}
\begin{table}
\centering
\begin{tabular}{lc}
\toprule
\textbf{Problema} & \textbf{Custo Anual (Estimativa)} \\
\midrule
Tempo perdido com manutenção & \$85 bilhões \\
Bugs e erros evitáveis & \$62 bilhões \\
Retrabalho por código ilegível & \$45 bilhões \\
Complexidade desnecessária & \$38 bilhões \\
\bottomrule
\end{tabular}
\end{table}
\vspace{0.5cm}
\footnotesize{Fonte: Estudos sobre produtividade em desenvolvimento de software (2023)}
\end{frame}

\begin{frame}{Caso Real: Knight Capital}
\begin{alertblock}{Um exemplo extremo}
Em 2012, a Knight Capital perdeu \$460 milhões em 45 minutos devido a:
\begin{itemize}
    \item Código não testado adequadamente
    \item Sistema de deploy falho
    \item Código legado não documentado
    \item Falta de práticas de segurança
\end{itemize}
\end{alertblock}

\begin{block}{Resultado}
\begin{itemize}
    \item Quase levou à falência da empresa
    \item Aquisição forçada por concorrente
    \item Multa de \$12 milhões
\end{itemize}
\end{block}
\end{frame}

\section{Benefícios das Boas Práticas}
\begin{frame}{Vantagens para Desenvolvedores}
\begin{columns}
\begin{column}{0.5\textwidth}
\begin{block}{Produtividade}
\begin{itemize}
    \item Menor tempo de debug
    \item Entendimento mais rápido
    \item Menos retrabalho
    \item Integração facilitada
\end{itemize}
\end{block}
\end{column}
\begin{column}{0.5\textwidth}
\begin{block}{Qualidade de Vida}
\begin{itemize}
    \item Menor estresse
    \item Maior satisfação
    \item Menor carga cognitiva
    \item Orgulho do trabalho
\end{itemize}
\end{block}
\end{column}
\end{columns}

\vspace{0.5cm}
\begin{center}
\begin{tikzpicture}
\node[draw, rounded corners, fill=blue!20] at (0,0) {
\footnotesize{\textbf{Desenvolvedores felizes escrevem código melhor}}
};
\end{tikzpicture}
\end{center}
\end{frame}

\begin{frame}{Vantagens para Empresas}
\begin{columns}
\begin{column}{0.5\textwidth}
\begin{block}{Econômicas}
\begin{itemize}
    \item Redução de custos
    \item Menor rotatividade
    \item Retorno mais rápido
    \item Menor risco legal
\end{itemize}
\end{block}
\end{column}
\begin{column}{0.5\textwidth}
\begin{block}{Estratégicas}
\begin{itemize}
    \item Competitividade
    \item Escalabilidade
    \item Adaptabilidade
    \item Reputação
\end{itemize}
\end{block}
\end{column}
\end{columns}

\vspace{0.5cm}
\begin{center}
% Ilustração de crescimento empresarial com TikZ
\begin{tikzpicture}[scale=0.6]
  % Eixos
  \draw[->] (0,0) -- (5,0) node[right] {\scriptsize{Tempo}};
  \draw[->] (0,0) -- (0,3) node[above] {\scriptsize{Valor}};
  
  % Curva de crescimento
  \draw[green!70!black, very thick] (0.5,0.3) .. controls (1.5,0.8) and (2.5,1.5) .. (4,2.5);
  
  % Pontos na curva
  \fill[blue] (0.5,0.3) circle (2pt) node[below] {\tiny{Início}};
  \fill[orange] (1.5,0.8) circle (2pt) node[below] {\tiny{Práticas}};
  \fill[green] (2.5,1.5) circle (2pt) node[below] {\tiny{CI/CD}};
  \fill[purple] (4,2.5) circle (2pt) node[below] {\tiny{Sucesso}};
  
  \node[below] at (2.5,-0.8) {\scriptsize{Crescimento com Boas Práticas}};
\end{tikzpicture}
\\\tiny{Fonte: Ilustração própria - Crescimento empresarial com boas práticas}
\end{center}
\end{frame}

\section{Princípios Fundamentais}
\begin{frame}{DRY: Don't Repeat Yourself}
\begin{block}{Princípio}
``Cada pedaço de conhecimento deve ter uma representação única, não ambígua e autoritativa dentro de um sistema.''
\end{block}

\begin{exampleblock}{Exemplo Prático}
\begin{itemize}
    \item \textbf{Ruim}: Mesmo código em 10 lugares diferentes
    \item \textbf{Bom}: Uma função bem nomeada reutilizável
    \item \textbf{Resultado}: Mudanças em um único lugar
\end{itemize}
\end{exampleblock}

\begin{center}
\begin{tikzpicture}
\node[draw, circle, fill=green!20] (one) at (0,0) {1};
\node[draw, circle, fill=red!20] (many) at (3,0) {Muitos};
\draw[->, thick] (many) to [bend right=30] node[above] {\scriptsize{DRY}} (one);
\end{tikzpicture}
\end{center}
\end{frame}

\begin{frame}{KISS: Keep It Simple, Stupid}
\begin{block}{Princípio}
``A simplicidade deve ser um objetivo chave no design, e a complexidade desnecessária deve ser evitada.''
\end{block}

\begin{columns}
\begin{column}{0.6\textwidth}
\begin{exampleblock}{Complexidade Acidental vs. Essencial}
\begin{itemize}
    \item \textbf{Essencial}: Complexidade inerente ao problema
    \item \textbf{Acidental}: Complexidade introduzida pela solução
    \item Boas práticas reduzem a acidental
\end{itemize}
\end{exampleblock}
\end{column}
\begin{column}{0.4\textwidth}
\centering
% Ilustração de complexidade com TikZ
\begin{tikzpicture}[scale=0.6]
  % Complexidade Essencial (lado esquerdo)
  \draw[green!70!black, thick] (-1.5,-1) rectangle (-0.5,1);
  \node[green!70!black] at (-1,1.3) {\tiny{\textbf{Essencial}}};
  
  % Elementos organizados
  \fill[green!60] (-1.3,0.5) circle (2pt);
  \fill[green!60] (-1,0) circle (2pt);
  \fill[green!60] (-0.7,-0.5) circle (2pt);
  \draw[green!70!black] (-1.3,0.5) -- (-1,0) -- (-0.7,-0.5);
  
  % Complexidade Acidental (lado direito)
  \draw[red!70!black, thick] (0.5,-1) rectangle (1.5,1);
  \node[red!70!black] at (1,1.3) {\tiny{\textbf{Acidental}}};
  
  % Elementos emaranhados
  \fill[red!60] (0.7,0.6) circle (1.5pt);
  \fill[red!60] (0.9,0.2) circle (1.5pt);
  \fill[red!60] (1.1,-0.1) circle (1.5pt);
  \fill[red!60] (1.3,0.4) circle (1.5pt);
  \fill[red!60] (0.8,-0.6) circle (1.5pt);
  
  % Conexões confusas
  \draw[red!70!black, bend left] (0.7,0.6) to (1.1,-0.1);
  \draw[red!70!black, bend right] (0.9,0.2) to (1.3,0.4);
  \draw[red!70!black, bend left] (1.1,-0.1) to (0.8,-0.6);
  
  % Seta
  \draw[->, blue, thick] (-0.3,0) -- (0.3,0);
  \node[blue, above] at (0,0.3) {\tiny{Reduzir}};
  
  \node[below] at (0,-1.5) {\scriptsize{Complexidade do Sistema}};
\end{tikzpicture}
\\\tiny{Fonte: Ilustração própria - Complexidade do sistema}
\end{column}
\end{columns}
\end{frame}

\begin{frame}{YAGNI: You Ain't Gonna Need It}
\begin{block}{Princípio}
``Implemente apenas funcionalidades que você realmente precisa, não as que você prevê que poderá precisar.''
\end{block}

\begin{alertblock}{Armadilha Comum}
\begin{itemize}
    \item Desperdício de tempo com features não usadas
    \item Complexidade desnecessária
    \item Dificuldade de manutenção
    \item Custo aumentado sem benefício
\end{itemize}
\end{alertblock}

\begin{center}
\begin{tikzpicture}
\node[draw, rounded corners, fill=yellow!20] at (0,0) {
\footnotesize{\textbf{Pergunte-se: Isso é necessário AGORA?}}
};
\end{tikzpicture}
\end{center}
\end{frame}

\section{CI/CD e Boas Práticas}

\begin{frame}{Benefícios do CI/CD}
\begin{columns}
\begin{column}{0.5\textwidth}
\begin{block}{Integração Contínua}
\begin{itemize}
    \item Testes automatizados
    \item Builds frequentes
    \item Detecção precoce de bugs
    \item Merge diário no mínimo
\end{itemize}
\end{block}
\end{column}
\begin{column}{0.5\textwidth}
\begin{block}{Entrega Contínua}
\begin{itemize}
    \item Deploy automatizado
    \item Sempre pronto para release
    \item Menor risco de entrega
    \item Feedback rápido dos clientes
\end{itemize}
\end{block}
\end{column}
\end{columns}

\vspace{0.5cm}
\begin{exampleblock}{Valor de Negócio}
\begin{itemize}
    \item \textbf{Comercial}: Receita direta e redução de custos
    \item \textbf{Mercado}: Atração de novos clientes e vantagem competitiva
    \item \textbf{Eficiência}: Otimização de processos e time-to-market
\end{itemize}
\end{exampleblock}
\end{frame}

\section{Impacto na Carreira}
\begin{frame}{Diferencial no Mercado de Trabalho}
\begin{block}{O que empregadores valorizam}
\begin{itemize}
    \item Capacidade de trabalhar em equipe
    \item Código de fácil manutenção e escalável
    \item Habilidade de documentar
    \item Entendimento de padrões e princípios
\end{itemize}
\end{block}

\begin{exampleblock}{Pesquisa com Tech Recruiters (2024)}
\begin{itemize}
    \item 87\% preferem desenvolvedores com boas práticas
    \item 76\% pagam até 30\% mais por esta habilidade
    \item 92\% consideram em processos seletivos
\end{itemize}
\end{exampleblock}
\end{frame}

\begin{frame}{Progressão na Carreira}
\begin{columns}
\begin{column}{0.5\textwidth}
\begin{block}{Junior → Pleno}
\begin{itemize}
    \item Escreve código que funciona
    \item Foca em tarefas individuais
    \item Precisa de supervisão
    \item Resolve problemas imediatos
\end{itemize}
\end{block}
\end{column}
\begin{column}{0.5\textwidth}
\begin{block}{Pleno → Sênior}
\begin{itemize}
    \item Escreve código de fácil manutenibilidade
    \item Pensa no sistema completo
    \item Orienta outros devs
    \item Antecipa problemas futuros
\end{itemize}
\end{block}
\end{column}
\end{columns}

\vspace{0.5cm}
\begin{center}
\begin{tikzpicture}
\node[draw, rounded corners, fill=blue!10] at (0,0) {
\footnotesize{\textbf{O uso correto de boas práticas separa os níveis de experiência entre os devs.}}
};
\end{tikzpicture}
\end{center}
\end{frame}

\section{Aplicação Prática}
\begin{frame}{Como Implementar no Dia a Dia}
\begin{block}{Passos Incrementais}
\begin{enumerate}
    \item \textbf{Revisar}: Analise seu próprio código criticamente
    \item \textbf{Identificar}: Detecte code smells e problemas
    \item \textbf{Refatorar}: Melhore gradualmente
    \item \textbf{Automatizar}: Use ferramentas de análise
    \item \textbf{Revisar}: Peça feedback constantemente
\end{enumerate}
\end{block}

\begin{exampleblock}{Ferramentas Úteis}
\begin{itemize}
    \item Linters (ESLint, Pylint, Checkstyle)
    \item Análise estática (SonarQube)
    \item Formatters (Prettier, Black)
    \item Testes automatizados
\end{itemize}
\end{exampleblock}
\end{frame}

\begin{frame}{Métricas de Sucesso}
\begin{columns}
\begin{column}{0.5\textwidth}
\begin{block}{Quantitativas}
\begin{itemize}
    \item ↓ Tempo de manutenção
    \item ↓ Bugs reportados
    \item ↓ Complexidade ciclomática
    \item ↑ Cobertura de testes
    \item ↑ Velocidade de desenvolvimento
\end{itemize}
\end{block}
\end{column}
\begin{column}{0.5\textwidth}
\begin{block}{Qualitativas}
\begin{itemize}
    \item ↑ Satisfação da equipe
    \item ↑ Confiança nas mudanças
    \item ↑ Clareza do código
    \item ↓ Estresse e sobrecarga
    \item ↑ Qualidade do produto
\end{itemize}
\end{block}
\end{column}
\end{columns}
\end{frame}

\section{Conclusão}
\begin{frame}{Por que Investir em Boas Práticas?}
\begin{block}{Resumo dos Benefícios}
\begin{itemize}
    \item \textbf{Econômico}: Reduz custos de desenvolvimento e manutenção
    \item \textbf{Técnico}: Aumenta qualidade e confiabilidade
    \item \textbf{Humano}: Melhora satisfação e colaboração
    \item \textbf{Profissional}: Diferencia no mercado de trabalho
\end{itemize}
\end{block}

\begin{center}
\begin{tikzpicture}
\node[draw, star, star points=5, fill=yellow!80!orange, inner sep=3pt] at (0,0) {};
\node at (0,-1) {\footnotesize{Investimento que sempre retorna}};
\end{tikzpicture}
\end{center}
\end{frame}

\begin{frame}
\centering
\Huge \textbf{Perguntas?}
\vspace{2cm}
\small \url{https://github.com/fmarquesfilho/bpp-2025-2}

\end{frame}

\end{document}
