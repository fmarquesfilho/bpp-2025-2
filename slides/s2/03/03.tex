% aula 3 - Código Limpo: Nomenclatura, Formatação e Estrutura

\documentclass[aspectratio=169]{beamer}
\UseRawInputEncoding
\usepackage[utf8]{inputenc}
\usepackage[brazil]{babel}
\usepackage{listings}
\usepackage{xcolor}
\usepackage{tikz}
\usepackage{fontawesome5}
\usepackage{graphicx}
\usepackage{adjustbox}
\usetikzlibrary{shapes.geometric, arrows, positioning, shadows}

% Tema moderno
\usetheme{metropolis}
\usecolortheme{default}

% Cores personalizadas
\definecolor{cleanblue}{RGB}{52, 152, 219}
\definecolor{cleangreen}{RGB}{46, 204, 113}
\definecolor{cleanred}{RGB}{231, 76, 60}
\definecolor{cleangray}{RGB}{149, 165, 166}
\definecolor{darkgray}{RGB}{44, 62, 80}

% Configuração do código
\lstdefinestyle{mystyle}{
    backgroundcolor=\color{gray!10},
    commentstyle=\color{cleangreen},
    keywordstyle=\color{cleanblue}\bfseries,
    numberstyle=\tiny\color{cleangray},
    stringstyle=\color{cleanred},
    basicstyle=\footnotesize\ttfamily,
    breakatwhitespace=false,
    breaklines=true,
    captionpos=b,
    keepspaces=true,
    numbers=left,
    numbersep=5pt,
    showspaces=false,
    showstringspaces=false,
    showtabs=false,
    tabsize=2,
    frame=single,
    rulecolor=\color{cleangray!30}
}
\lstset{style=mystyle}

% Reduzir espaçamento entre itens
\setlength{\itemsep}{2pt}
\setlength{\parskip}{2pt}

% Informações do documento
\title{Código Limpo: Princípios e Práticas}
\subtitle{Boas Práticas de Programação - BPP 2025.2}
\author{Prof. Fernando Marques Filho}
\institute{Universidade Federal do Rio Grande do Norte}
\date{\today}

\begin{document}

% Slide de título
\frame{\titlepage}

% Sumário
\begin{frame}
\frametitle{Agenda}
\tableofcontents
\end{frame}

\section{Introdução ao Código Limpo}

\begin{frame}
\frametitle{O que é Código Limpo?}
\begin{center}
\begin{tikzpicture}[scale=0.7]
    \node[draw, rectangle, fill=cleanblue!20, minimum width=7cm, minimum height=2.5cm, rounded corners=10pt] (main) at (0,0) {};
    \node[text width=6cm, align=center] at (0,0.3) {\Large\textbf{Código Limpo}};
    \node[text width=6cm, align=center] at (0,-0.2) {É um código};
    \node[text width=6cm, align=center] at (0,-0.7) {\textcolor{cleanblue}{\textbf{fácil de entender}} e \textcolor{cleangreen}{\textbf{fácil de alterar}}};
\end{tikzpicture}
\end{center}

\vspace{0.3cm}
\begin{center}
\footnotesize
\textit{"Qualquer tolo consegue escrever código que um computador entende.\\
Bons programadores escrevem código que humanos podem entender."}\\
\vspace{0.1cm}
\textbf{— Martin Fowler}
\end{center}
\end{frame}

\begin{frame}
\frametitle{Origem e Motivação}
\begin{columns}
\begin{column}{0.6\textwidth}
\footnotesize
\begin{itemize}
    \item \textbf{Criado por:} Robert C. Martin (Uncle Bob)
    \item \textbf{Motivação:} Combater o débito técnico
    \item \textbf{Objetivo:} Software sustentável e de qualidade
\end{itemize}

\vspace{0.3cm}
\begin{block}{Débito Técnico}
\scriptsize
Custo implícito de uma implementação pensada apenas no agora, em vez de usar uma abordagem de melhor qualidade.
\end{block}
\end{column}

\begin{column}{0.4\textwidth}
\begin{center}
\begin{tikzpicture}[scale=0.8]
    \draw[fill=cleanred!20, draw=cleanred, thick] (0,0) rectangle (2.5,1.7);
    \node at (1.25,1.3) {\faExclamationTriangle};
    \node[text width=2.3cm, align=center, font=\scriptsize] at (1.25,0.7) {Débito Técnico};
    \node[text width=2.3cm, align=center, font=\tiny] at (1.25,0.3) {Soluções rápidas geram problemas futuros};
\end{tikzpicture}
\end{center}
\end{column}
\end{columns}
\end{frame}

\section{Nomes Significativos}

\begin{frame}
\frametitle{Nomes Significativos}
\framesubtitle{Use nomes que revelem o propósito}

\begin{columns}
\begin{column}{0.5\textwidth}
\begin{block}{\textcolor{cleanred}{\faTimes} Ruim}
\scriptsize
\lstinputlisting[language=Java, basicstyle=\scriptsize\ttfamily]{examples/bad_names.java}
\end{block}
\end{column}

\begin{column}{0.5\textwidth}
\begin{block}{\textcolor{cleangreen}{\faCheck} Bom}
\scriptsize
\lstinputlisting[language=Java, basicstyle=\scriptsize\ttfamily]{examples/good_names.java}
\end{block}
\end{column}
\end{columns}

\vspace{0.3cm}
\begin{alertblock}{Regra de Ouro}
\centering
\footnotesize
O nome deve responder: \textbf{Por que existe? O que faz? Como é usado?}
\end{alertblock}
\end{frame}

\begin{frame}
\frametitle{Nomes Pronunciáveis e Buscáveis}

\begin{columns}
\begin{column}{0.5\textwidth}
\begin{block}{\textcolor{cleanred}{\faTimes} Evite}
\scriptsize
\lstinputlisting[language=Java, basicstyle=\scriptsize\ttfamily]{examples/bad_searchable.java}
\end{block}
\end{column}

\begin{column}{0.5\textwidth}
\begin{block}{\textcolor{cleangreen}{\faCheck} Prefira}
\scriptsize
\lstinputlisting[language=Java, basicstyle=\scriptsize\ttfamily]{examples/good_searchable.java}
\end{block}
\end{column}
\end{columns}

\vspace{0.3cm}
\footnotesize
\begin{itemize}
    \item \textbf{Pronunciáveis:} Facilita comunicação entre equipe
    \item \textbf{Buscáveis:} Permite encontrar rapidamente no código
    \item \textbf{Sem prefixos:} Evite notações húngaras (strNome, intIdade)
\end{itemize}
\end{frame}

\begin{frame}
\frametitle{Classes e Métodos: Nomenclatura}

\begin{columns}
\begin{column}{0.5\textwidth}
\begin{block}{\textcolor{cleanred}{\faTimes} Classes Genéricas}
\scriptsize
\lstinputlisting[language=Java, basicstyle=\scriptsize\ttfamily]{examples/bad_class_names.java}
\end{block}
\end{column}

\begin{column}{0.5\textwidth}
\begin{block}{\textcolor{cleangreen}{\faCheck} Classes Específicas}
\scriptsize
\lstinputlisting[language=Java, basicstyle=\scriptsize\ttfamily]{examples/good_class_names.java}
\end{block}
\end{column}
\end{columns}

\vspace{0.3cm}
\footnotesize
\begin{itemize}
    \item \textbf{Classes:} Substantivos ou frases nominais
    \item \textbf{Métodos:} Verbos ou frases verbais
    \item \textbf{Evite:} Manager, Processor, Data, Info
\end{itemize}
\end{frame}

\section{Funções}

\begin{frame}
\frametitle{Funções: Pequenas e Focadas}
\framesubtitle{Princípio da Responsabilidade Única}

\begin{center}
\begin{tikzpicture}[scale=0.8]
    \draw[fill=cleanred!20, draw=cleanred, thick] (0,0) rectangle (3.5,2.5);
    \node[text width=3cm, align=center, font=\scriptsize] at (1.75,1.2) {\textbf{Função Grande}\\
    \tiny • Múltiplas responsabilidades\\
    • Difícil de testar\\
    • Difícil de entender};
    
    \draw[->, very thick] (3.8,1.2) -- (4.8,1.2);
    
    \draw[fill=cleangreen!20, draw=cleangreen, thick] (5.2,0) rectangle (6.7,0.8);
    \node[text width=1.3cm, align=center, font=\tiny] at (5.95,0.4) {Função 1};
    
    \draw[fill=cleangreen!20, draw=cleangreen, thick] (5.2,1) rectangle (6.7,1.8);
    \node[text width=1.3cm, align=center, font=\tiny] at (5.95,1.4) {Função 2};
    
    \draw[fill=cleangreen!20, draw=cleangreen, thick] (5.2,2) rectangle (6.7,2.8);
    \node[text width=1.3cm, align=center, font=\tiny] at (5.95,2.4) {Função 3};
\end{tikzpicture}
\end{center}

\vspace{0.2cm}
\begin{alertblock}{Regra}
\centering
\footnotesize
Uma função deve fazer uma coisa, fazê-la bem e fazer apenas ela.
\end{alertblock}
\end{frame}

\begin{frame}
\frametitle{Exemplo: Refatoração de Função}

\begin{block}{\textcolor{cleanred}{\faTimes} Função com Múltiplas Responsabilidades}
\scriptsize
\lstinputlisting[language=Python, basicstyle=\scriptsize\ttfamily, firstline=1, lastline=6]{examples/bad_function.py}
\end{block}

\begin{block}{\textcolor{cleangreen}{\faCheck} Funções Especializadas}
\scriptsize
\lstinputlisting[language=Python, basicstyle=\scriptsize\ttfamily, firstline=1, lastline=10]{examples/good_functions.py}
\end{block}
\end{frame}

\begin{frame}
\frametitle{Argumentos de Função}
\framesubtitle{Minimize o número de parâmetros}

\begin{center}
\begin{tikzpicture}[scale=0.8]
    \node[draw, circle, fill=cleangreen!30, font=\scriptsize] (zero) at (0,1.5) {0};
    \node[draw, circle, fill=cleangreen!20, font=\scriptsize] (one) at (1.8,1.5) {1};
    \node[draw, circle, fill=yellow!30, font=\scriptsize] (two) at (3.6,1.5) {2};
    \node[draw, circle, fill=orange!30, font=\scriptsize] (three) at (5.4,1.5) {3};
    \node[draw, circle, fill=cleanred!30, font=\scriptsize] (more) at (7.2,1.5) {3+};
    
    \node[below=0.2cm of zero, font=\tiny] {\small Ideal};
    \node[below=0.2cm of one, font=\tiny] {\small Bom};
    \node[below=0.2cm of two, font=\tiny] {\small Aceitável};
    \node[below=0.2cm of three, font=\tiny] {\small Evitar};
    \node[below=0.2cm of more, font=\tiny] {\small Refatorar};
\end{tikzpicture}
\end{center}

\vspace{0.3cm}
\begin{columns}
\begin{column}{0.5\textwidth}
\begin{block}{\textcolor{cleanred}{\faTimes} Muitos Parâmetros}
\scriptsize
\lstinputlisting[language=Java, basicstyle=\scriptsize\ttfamily]{examples/many_params.java}
\end{block}
\end{column}

\begin{column}{0.5\textwidth}
\begin{block}{\textcolor{cleangreen}{\faCheck} Objeto de Parâmetro}
\scriptsize
\lstinputlisting[language=Java, basicstyle=\scriptsize\ttfamily]{examples/param_object.java}
\end{block}
\end{column}
\end{columns}
\end{frame}

\begin{frame}
\frametitle{Efeitos Colaterais em Funções}

\begin{alertblock}{Problema}
\footnotesize
Funções que fazem mais do que prometem em seu nome causam efeitos colaterais inesperados.
\end{alertblock}

\vspace{0.3cm}
\begin{columns}
\begin{column}{0.5\textwidth}
\begin{block}{\textcolor{cleanred}{\faTimes} Com Efeito Colateral}
\scriptsize
\lstinputlisting[language=Java, basicstyle=\scriptsize\ttfamily]{examples/side_effect.java}
\end{block}
\end{column}

\begin{column}{0.5\textwidth}
\begin{block}{\textcolor{cleangreen}{\faCheck} Sem Efeito Colateral}
\scriptsize
\lstinputlisting[language=Java, basicstyle=\scriptsize\ttfamily]{examples/no_side_effect.java}
\end{block}
\end{column}
\end{columns}
\end{frame}

\section{Comentários}

\begin{frame}
\frametitle{Comentários: Quando e Como}
\framesubtitle{Comentários são sintomas, não soluções}

\begin{center}
\begin{tikzpicture}[scale=0.9]
    \node[draw, rectangle, fill=cleanblue!20, minimum width=7cm, minimum height=1.5cm, rounded corners=5pt] at (0,0) {
        \begin{minipage}{6.5cm}
        \centering
        \footnotesize
        \textit{"A necessidade de comentários muitas vezes indica\\que o código não está claro o suficiente"}\\
        \textbf{— Uncle Bob}
        \end{minipage}
    };
\end{tikzpicture}
\end{center}

\vspace{0.3cm}
\begin{columns}
\begin{column}{0.3\textwidth}
\begin{block}{\textcolor{cleangreen}{\faCheck} Bons Comentários}
\scriptsize
\begin{itemize}
    \item Explicação de intenções
    \item Esclarecimentos
    \item Avisos de consequências
    \item TODOs
\end{itemize}
\end{block}
\end{column}

\begin{column}{0.3\textwidth}
\begin{block}{\textcolor{cleanred}{\faTimes} Maus Comentários}
\scriptsize
\begin{itemize}
    \item Murmúrios
    \item Redundantes
    \item Enganosos
    \item Código comentado
\end{itemize}
\end{block}
\end{column}

\begin{column}{0.3\textwidth}
\begin{block}{\faInfoCircle\ Regra}
\scriptsize
Escreva código autoexplicativo primeiro. Use comentários apenas quando necessário.
\end{block}
\end{column}
\end{columns}
\end{frame}

\section{Conclusão}

\begin{frame}
\frametitle{Resumo dos Princípios}

\begin{center}
\begin{tikzpicture}[node distance=1.5cm, scale=0.8]
    \node[rectangle, draw, fill=cleanblue!20, text width=1.8cm, align=center, font=\scriptsize] (nomes) {Nomes\\Significativos};
    \node[rectangle, draw, fill=cleangreen!20, text width=1.8cm, align=center, font=\scriptsize, right=of nomes] (funcoes) {Funções\\Pequenas};
    \node[rectangle, draw, fill=yellow!20, text width=1.8cm, align=center, font=\scriptsize, right=of funcoes] (comentarios) {Comentários\\Úteis};
    
    \node[rectangle, draw, fill=orange!20, text width=1.8cm, align=center, font=\scriptsize, below=0.8cm of nomes] (formatacao) {Formatação\\Consistente};
    \node[rectangle, draw, fill=purple!20, text width=1.8cm, align=center, font=\scriptsize, right=of formatacao] (objetos) {Objetos\\Bem Definidos};
    
    % Setas conectoras
    \draw[->, thick] (nomes) -- (funcoes);
    \draw[->, thick] (funcoes) -- (comentarios);
    \draw[->, thick] (nomes) -- (formatacao);
    \draw[->, thick] (formatacao) -- (objetos);
    \draw[->, thick] (funcoes) -- (objetos);
\end{tikzpicture}
\end{center}

\vspace{0.3cm}
\begin{alertblock}{Lembre-se}
\centering
\footnotesize
Código limpo não é escrito de uma vez. É refinado continuamente.
\end{alertblock}
\end{frame}

\begin{frame}
\frametitle{Próximos Passos}

\footnotesize
\begin{enumerate}
    \item \textbf{Pratique:} Aplique esses princípios no seu código diário
    \item \textbf{Refatore:} Melhore código existente gradualmente
    \item \textbf{Code Review:} Use esses critérios para avaliar código
    \item \textbf{Ferramentas:} Utilize analisadores estáticos (próxima aula)
\end{enumerate}

\vspace{1cm}
\begin{center}
\textbf{Dúvidas?}
\end{center}
\end{frame}

\begin{frame}
\frametitle{Referências}

\scriptsize
\begin{itemize}
    \item Martin, Robert C. \textbf{Clean Code: A Handbook of Agile Software Craftsmanship}. Prentice Hall, 2008.
    \item Fowler, Martin. \textbf{Refactoring: Improving the Design of Existing Code}. Addison-Wesley, 2019.
    \item Catálogo de Code Smells: \url{https://luzkan.github.io/smells/}
    \item Repositório do curso: \url{https://github.com/fmarquesfilho/bpp-2025-2}
\end{itemize}

\vspace{0.5cm}
\begin{center}
\Large
Obrigado pela atenção!\\
\vspace{0.3cm}
\normalsize
\texttt{fernando@dimap.ufrn.br}
\end{center}
\end{frame}

\end{document}