\documentclass[aspectratio=169]{beamer}
\usepackage[utf8]{inputenc}
\usepackage[brazil]{babel}
\usepackage{listings}
\usepackage{xcolor}
\usepackage{tikz}
\usepackage{fontawesome5}
\usepackage{graphicx}
\usetikzlibrary{shapes.geometric, arrows, positioning, shadows}

% Tema moderno
\usetheme{metropolis}
\usecolortheme{default}

% Cores personalizadas
\definecolor{cleanblue}{RGB}{52, 152, 219}
\definecolor{cleangreen}{RGB}{46, 204, 113}
\definecolor{cleanred}{RGB}{231, 76, 60}
\definecolor{cleangray}{RGB}{149, 165, 166}
\definecolor{darkgray}{RGB}{44, 62, 80}

% Configuração do código
\lstdefinestyle{mystyle}{
    backgroundcolor=\color{gray!10},
    commentstyle=\color{cleangreen},
    keywordstyle=\color{cleanblue}\bfseries,
    numberstyle=\tiny\color{cleangray},
    stringstyle=\color{cleanred},
    basicstyle=\footnotesize\ttfamily,
    breakatwhitespace=false,
    breaklines=true,
    captionpos=b,
    keepspaces=true,
    numbers=left,
    numbersep=5pt,
    showspaces=false,
    showstringspaces=false,
    showtabs=false,
    tabsize=2,
    frame=single,
    rulecolor=\color{cleangray!30}
}
\lstset{style=mystyle}

% Informações do documento
\title{Código Limpo: Princípios e Práticas}
\subtitle{Boas Práticas de Programação - BPP 2025.2}
\author{Prof. Fernando Marques Filho}
\institute{Universidade Federal do Rio Grande do Norte}
\date{\today}

\begin{document}

% Slide de título
\frame{\titlepage}

% Sumário
\begin{frame}
\frametitle{Agenda}
\tableofcontents
\end{frame}

\section{Introdução ao Código Limpo}

\begin{frame}
\frametitle{O que é Código Limpo?}
\begin{center}
\begin{tikzpicture}[scale=0.8]
    \node[draw, rectangle, fill=cleanblue!20, minimum width=8cm, minimum height=3cm, rounded corners=10pt] (main) at (0,0) {};
    \node[text width=7cm, align=center] at (0,0.5) {\Large\textbf{Código Limpo}};
    \node[text width=7cm, align=center] at (0,0) {É um código};
    \node[text width=7cm, align=center] at (0,-0.5) {\textcolor{cleanblue}{\textbf{fácil de entender}} e \textcolor{cleangreen}{\textbf{fácil de alterar}}};
\end{tikzpicture}
\end{center}

\vspace{0.5cm}
\begin{center}
\textit{"Qualquer tolo consegue escrever código que um computador entende.\\
Bons programadores escrevem código que humanos podem entender."}\\
\vspace{0.2cm}
\textbf{— Martin Fowler}
\end{center}
\end{frame}

\begin{frame}
\frametitle{Origem e Motivação}
\begin{columns}
\begin{column}{0.6\textwidth}
\begin{itemize}
    \item \textbf{Criado por:} Robert C. Martin (Uncle Bob)
    \item \textbf{Motivação:} Combater o débito técnico
    \item \textbf{Objetivo:} Software sustentável e de qualidade
\end{itemize}

\vspace{0.5cm}
\begin{block}{Débito Técnico}
Custo implícito de uma implementação pensada apenas no agora, em vez de usar uma abordagem de melhor qualidade.
\end{block}
\end{column}

\begin{column}{0.4\textwidth}
\begin{center}
\begin{tikzpicture}
    \draw[fill=cleanred!20, draw=cleanred, thick] (0,0) rectangle (3,2);
    \node at (1.5,1.5) {\faExclamationTriangle};
    \node[text width=2.8cm, align=center] at (1.5,0.8) {\small Débito Técnico};
    \node[text width=2.8cm, align=center] at (1.5,0.3) {\tiny Soluções rápidas geram problemas futuros};
\end{tikzpicture}
\end{center}
\end{column}
\end{columns}
\end{frame}

\begin{frame}
\frametitle{Princípios do Código Limpo}
\begin{center}
\begin{tikzpicture}[node distance=1.5cm]
    % Nó central
    \node[circle, draw, fill=cleanblue!30, minimum size=2cm] (center) {\textbf{Clean\\Code}};
    
    % Princípios ao redor
    \node[rectangle, draw, fill=cleangreen!20, above left=of center] (eficiente) {\small Eficiente};
    \node[rectangle, draw, fill=cleangreen!20, above=of center] (simples) {\small Simples};
    \node[rectangle, draw, fill=cleangreen!20, above right=of center] (objetivo) {\small Objetivo};
    \node[rectangle, draw, fill=cleangreen!20, right=of center] (elegante) {\small Elegante};
    \node[rectangle, draw, fill=cleangreen!20, below right=of center] (leitura) {\small Fácil Leitura};
    \node[rectangle, draw, fill=cleangreen!20, below=of center] (manutencao) {\small Fácil Manutenção};
    \node[rectangle, draw, fill=cleangreen!20, below left=of center] (testes) {\small Testável};
    \node[rectangle, draw, fill=cleangreen!20, left=of center] (dependencias) {\small Poucas Dependências};
    
    % Linhas conectoras
    \draw[->] (center) -- (eficiente);
    \draw[->] (center) -- (simples);
    \draw[->] (center) -- (objetivo);
    \draw[->] (center) -- (elegante);
    \draw[->] (center) -- (leitura);
    \draw[->] (center) -- (manutencao);
    \draw[->] (center) -- (testes);
    \draw[->] (center) -- (dependencias);
\end{tikzpicture}
\end{center}
\end{frame}

\section{Nomes Significativos}

\begin{frame}
\frametitle{Nomes Significativos}
\framesubtitle{Use nomes que revelem o propósito}

\begin{columns}
\begin{column}{0.5\textwidth}
\begin{block}{\textcolor{cleanred}{\faTimes} Ruim}
\lstinputlisting[language=Java]{examples/bad_names.java}
\end{block}
\end{column}

\begin{column}{0.5\textwidth}
\begin{block}{\textcolor{cleangreen}{\faCheck} Bom}
\lstinputlisting[language=Java]{examples/good_names.java}
\end{block}
\end{column}
\end{columns}

\vspace{0.5cm}
\begin{alertblock}{Regra de Ouro}
\centering
O nome deve responder: \textbf{Por que existe? O que faz? Como é usado?}
\end{alertblock}
\end{frame}

\begin{frame}
\frametitle{Nomes Pronunciáveis e Buscáveis}

\begin{columns}
\begin{column}{0.5\textwidth}
\begin{block}{\textcolor{cleanred}{\faTimes} Evite}
\lstinputlisting[language=Java]{examples/bad_searchable.java}
\end{block}
\end{column}

\begin{column}{0.5\textwidth}
\begin{block}{\textcolor{cleangreen}{\faCheck} Prefira}
\lstinputlisting[language=Java]{examples/good_searchable.java}
\end{block}
\end{column}
\end{columns}

\vspace{0.5cm}
\begin{itemize}
    \item \textbf{Pronunciáveis:} Facilita comunicação entre equipe
    \item \textbf{Buscáveis:} Permite encontrar rapidamente no código
    \item \textbf{Sem prefixos:} Evite notações húngaras (strNome, intIdade)
\end{itemize}
\end{frame}

\begin{frame}
\frametitle{Classes e Métodos: Nomenclatura}

\begin{columns}
\begin{column}{0.5\textwidth}
\begin{block}{\textcolor{cleanred}{\faTimes} Classes Genéricas}
\lstinputlisting[language=Java]{examples/bad_class_names.java}
\end{block}
\end{column}

\begin{column}{0.5\textwidth}
\begin{block}{\textcolor{cleangreen}{\faCheck} Classes Específicas}
\lstinputlisting[language=Java]{examples/good_class_names.java}
\end{block}
\end{column}
\end{columns}

\vspace{0.5cm}
\begin{itemize}
    \item \textbf{Classes:} Substantivos ou frases nominais
    \item \textbf{Métodos:} Verbos ou frases verbais
    \item \textbf{Evite:} Manager, Processor, Data, Info
\end{itemize}
\end{frame}

\section{Funções}

\begin{frame}
\frametitle{Funções: Pequenas e Focadas}
\framesubtitle{Princípio da Responsabilidade Única}

\begin{center}
\begin{tikzpicture}
    \draw[fill=cleanred!20, draw=cleanred, thick] (0,0) rectangle (4,3);
    \node[text width=3.5cm, align=center] at (2,1.5) {\textbf{Função Grande}\\
    \small • Múltiplas responsabilidades\\
    • Difícil de testar\\
    • Difícil de entender};
    
    \draw[->, very thick] (4.5,1.5) -- (5.5,1.5);
    
    \draw[fill=cleangreen!20, draw=cleangreen, thick] (6,0) rectangle (8,1);
    \node[text width=1.8cm, align=center] at (7,0.5) {\small Função 1};
    
    \draw[fill=cleangreen!20, draw=cleangreen, thick] (6,1.2) rectangle (8,2.2);
    \node[text width=1.8cm, align=center] at (7,1.7) {\small Função 2};
    
    \draw[fill=cleangreen!20, draw=cleangreen, thick] (6,2.4) rectangle (8,3.4);
    \node[text width=1.8cm, align=center] at (7,2.9) {\small Função 3};
\end{tikzpicture}
\end{center}

\begin{alertblock}{Regra}
Uma função deve fazer uma coisa, fazê-la bem e fazer apenas ela.
\end{alertblock}
\end{frame}

\begin{frame}
\frametitle{Exemplo: Refatoração de Função}

\begin{block}{\textcolor{cleanred}{\faTimes} Função com Múltiplas Responsabilidades}
\lstinputlisting[language=Python, firstline=1, lastline=8]{examples/bad_function.py}
\end{block}

\begin{block}{\textcolor{cleangreen}{\faCheck} Funções Especializadas}
\lstinputlisting[language=Python, firstline=1, lastline=12]{examples/good_functions.py}
\end{block}
\end{frame}

\begin{frame}
\frametitle{Argumentos de Função}
\framesubtitle{Minimize o número de parâmetros}

\begin{center}
\begin{tikzpicture}
    \node[draw, circle, fill=cleangreen!30] (zero) at (0,2) {0};
    \node[draw, circle, fill=cleangreen!20] (one) at (2,2) {1};
    \node[draw, circle, fill=yellow!30] (two) at (4,2) {2};
    \node[draw, circle, fill=orange!30] (three) at (6,2) {3};
    \node[draw, circle, fill=cleanred!30] (more) at (8,2) {3+};
    
    \node[below=0.3cm of zero] {\small Ideal};
    \node[below=0.3cm of one] {\small Bom};
    \node[below=0.3cm of two] {\small Aceitável};
    \node[below=0.3cm of three] {\small Evitar};
    \node[below=0.3cm of more] {\small Refatorar};
\end{tikzpicture}
\end{center}

\vspace{0.5cm}
\begin{columns}
\begin{column}{0.5\textwidth}
\begin{block}{\textcolor{cleanred}{\faTimes} Muitos Parâmetros}
\lstinputlisting[language=Java]{examples/many_params.java}
\end{block}
\end{column}

\begin{column}{0.5\textwidth}
\begin{block}{\textcolor{cleangreen}{\faCheck} Objeto de Parâmetro}
\lstinputlisting[language=Java]{examples/param_object.java}
\end{block}
\end{column}
\end{columns}
\end{frame}

\begin{frame}
\frametitle{Efeitos Colaterais em Funções}

\begin{alertblock}{Problema}
Funções que fazem mais do que prometem em seu nome causam efeitos colaterais inesperados.
\end{alertblock}

\vspace{0.5cm}
\begin{columns}
\begin{column}{0.5\textwidth}
\begin{block}{\textcolor{cleanred}{\faTimes} Com Efeito Colateral}
\lstinputlisting[language=Java]{examples/side_effect.java}
\end{block}
\end{column}

\begin{column}{0.5\textwidth}
\begin{block}{\textcolor{cleangreen}{\faCheck} Sem Efeito Colateral}
\lstinputlisting[language=Java]{examples/no_side_effect.java}
\end{block}
\end{column}
\end{columns}
\end{frame}

\section{Comentários}

\begin{frame}
\frametitle{Comentários: Quando e Como}
\framesubtitle{Comentários são sintomas, não soluções}

\begin{center}
\begin{tikzpicture}
    \node[draw, rectangle, fill=cleanblue!20, minimum width=8cm, minimum height=2cm, rounded corners=5pt] at (0,0) {
        \begin{minipage}{7.5cm}
        \centering
        \textit{"A necessidade de comentários muitas vezes indica\\que o código não está claro o suficiente"}\\
        \textbf{— Uncle Bob}
        \end{minipage}
    };
\end{tikzpicture}
\end{center}

\vspace{0.5cm}
\begin{columns}
\begin{column}{0.3\textwidth}
\begin{block}{\textcolor{cleangreen}{\faCheck} Bons Comentários}
\begin{itemize}
    \item Explicação de intenções
    \item Esclarecimentos
    \item Avisos de consequências
    \item TODOs
\end{itemize}
\end{block}
\end{column}

\begin{column}{0.3\textwidth}
\begin{block}{\textcolor{cleanred}{\faTimes} Maus Comentários}
\begin{itemize}
    \item Murmúrios
    \item Redundantes
    \item Enganosos
    \item Código comentado
\end{itemize}
\end{block}
\end{column}

\begin{column}{0.3\textwidth}
\begin{block}{\faInfoCircle} Regra}
Escreva código autoexplicativo primeiro. Use comentários apenas quando necessário.
\end{block}
\end{column}
\end{columns}
\end{frame}

\begin{frame}
\frametitle{Exemplos de Comentários}

\begin{columns}
\begin{column}{0.5\textwidth}
\begin{block}{\textcolor{cleanred}{\faTimes} Comentário Redundante}
\lstinputlisting[language=Java]{examples/bad_comment.java}
\end{block}

\vspace{0.3cm}
\begin{block}{\textcolor{cleangreen}{\faCheck} Comentário de Intenção}
\lstinputlisting[language=Java]{examples/good_comment.java}
\end{block}
\end{column}

\begin{column}{0.5\textwidth}
\begin{block}{\textcolor{cleangreen}{\faCheck} Comentário de Aviso}
\lstinputlisting[language=Java]{examples/warning_comment.java}
\end{block}

\vspace{0.3cm}
\begin{block}{\textcolor{cleangreen}{\faCheck} Melhor: Código Claro}
\lstinputlisting[language=Java]{examples/self_documenting.java}
\end{block}
\end{column}
\end{columns}
\end{frame}

\section{Formatação}

\begin{frame}
\frametitle{Formatação: A Importância da Apresentação}

\begin{center}
\begin{tikzpicture}
    \node[draw, rectangle, fill=cleanblue!20, minimum width=10cm, minimum height=2.5cm, rounded corners=10pt] at (0,0) {
        \begin{minipage}{9cm}
        \centering
        \Large\textbf{Formatação não é luxo, é necessidade}\\
        \vspace{0.3cm}
        \normalsize Um código bem formatado é mais fácil de entender\\
        porque permite focar no conteúdo, não na apresentação
        \end{minipage}
    };
\end{tikzpicture}
\end{center}

\vspace{0.5cm}
\begin{columns}
\begin{column}{0.5\textwidth}
\begin{itemize}
    \item \textbf{Formatação Vertical:} Como organizar o código de cima para baixo
    \item \textbf{Formatação Horizontal:} Como organizar cada linha
    \item \textbf{Consistência:} Seguir padrões definidos
\end{itemize}
\end{column}

\begin{column}{0.5\textwidth}
\begin{center}
\begin{tikzpicture}[scale=0.8]
    \draw[fill=cleangray!10] (0,0) rectangle (3,4);
    \draw[thick] (0.2,3.5) -- (2.8,3.5);
    \draw[thick] (0.2,3.0) -- (2.8,3.0);
    \draw[thick] (0.2,2.5) -- (1.5,2.5);
    \draw[thick] (0.2,2.0) -- (2.8,2.0);
    \draw[thick] (0.2,1.5) -- (2.8,1.5);
    \draw[thick] (0.2,1.0) -- (1.8,1.0);
    \draw[thick] (0.2,0.5) -- (2.8,0.5);
    \node[below] at (1.5,0) {\small Código bem formatado};
\end{tikzpicture}
\end{center}
\end{column}
\end{columns}
\end{frame}

\begin{frame}
\frametitle{Formatação Vertical: Metáfora do Jornal}

\begin{columns}
\begin{column}{0.6\textwidth}
\begin{enumerate}
    \item \textbf{Título:} Nome da classe/arquivo
    \item \textbf{Primeiro parágrafo:} Visão geral de alto nível
    \item \textbf{Parágrafos seguintes:} Detalhes específicos
    \item \textbf{Seções:} Agrupamento por conceitos relacionados
\end{enumerate}

\vspace{0.5cm}
\begin{block}{Diretrizes}
\begin{itemize}
    \item Espaçamento entre conceitos
    \item Declarações relacionadas próximas
    \item Ordenação vertical (dependências apontam para baixo)
\end{itemize}
\end{block}
\end{column}

\begin{column}{0.4\textwidth}
\begin{center}
\begin{tikzpicture}[scale=0.7]
    % Simulando layout de jornal
    \draw[thick] (0,0) rectangle (3,5);
    \draw[fill=darkgray] (0.2,4.2) rectangle (2.8,4.8);
    \node[white, font=\tiny] at (1.5,4.5) {TÍTULO PRINCIPAL};
    
    \draw[thick] (0.2,3.5) -- (2.8,3.5);
    \draw[thick] (0.2,3.3) -- (2.8,3.3);
    \draw[thick] (0.2,3.1) -- (2.0,3.1);
    
    \draw[thick] (0.2,2.7) -- (2.8,2.7);
    \draw[thick] (0.2,2.5) -- (2.8,2.5);
    \draw[thick] (0.2,2.3) -- (1.8,2.3);
    
    \draw[thick] (0.2,1.9) -- (2.8,1.9);
    \draw[thick] (0.2,1.7) -- (2.8,1.7);
    \draw[thick] (0.2,1.5) -- (2.3,1.5);
    
    \node[below] at (1.5,0) {\tiny Estrutura hierárquica};
\end{tikzpicture}
\end{center}
\end{column}
\end{columns}
\end{frame}

\begin{frame}
\frametitle{Formatação Horizontal e Indentação}

\begin{columns}
\begin{column}{0.5\textwidth}
\begin{block}{\textcolor{cleanred}{\faTimes} Formatação Ruim}
\lstinputlisting[language=Java]{examples/bad_formatting.java}
\end{block}
\end{column}

\begin{column}{0.5\textwidth}
\begin{block}{\textcolor{cleangreen}{\faCheck} Formatação Boa}
\lstinputlisting[language=Java]{examples/good_formatting.java}
\end{block}
\end{column}
\end{columns}

\vspace{0.5cm}
\begin{itemize}
    \item \textbf{Indentação:} Reflete a hierarquia lógica
    \item \textbf{Espaçamento:} Separa operadores e conceitos
    \item \textbf{Alinhamento:} Melhora a legibilidade
\end{itemize}
\end{frame}

\section{Objetos e Estruturas de Dados}

\begin{frame}
\frametitle{Objetos vs. Estruturas de Dados}

\begin{columns}
\begin{column}{0.5\textwidth}
\begin{block}{Estruturas de Dados}
\begin{itemize}
    \item Expõem dados
    \item Não têm comportamento significativo
    \item Fácil adicionar novas funções
    \item Difícil adicionar novos tipos
\end{itemize}
\end{block}

\begin{center}
\begin{tikzpicture}[scale=0.8]
    \draw[fill=cleanblue!20] (0,0) rectangle (2.5,2);
    \node[text width=2.3cm, align=center] at (1.25,1.5) {\small \textbf{Estrutura}};
    \node[text width=2.3cm, align=center] at (1.25,1) {\tiny + dados públicos};
    \node[text width=2.3cm, align=center] at (1.25,0.5) {\tiny - comportamento};
\end{tikzpicture}
\end{center}
\end{column}

\begin{column}{0.5\textwidth}
\begin{block}{Objetos}
\begin{itemize}
    \item Escondem dados
    \item Expõem comportamento
    \item Fácil adicionar novos tipos
    \item Difícil adicionar novas funções
\end{itemize}
\end{block}

\begin{center}
\begin{tikzpicture}[scale=0.8]
    \draw[fill=cleangreen!20] (0,0) rectangle (2.5,2);
    \node[text width=2.3cm, align=center] at (1.25,1.5) {\small \textbf{Objeto}};
    \node[text width=2.3cm, align=center] at (1.25,1) {\tiny - dados privados};
    \node[text width=2.3cm, align=center] at (1.25,0.5) {\tiny + comportamento};
\end{tikzpicture}
\end{center}
\end{column}
\end{columns}

\vspace{0.5cm}
\begin{alertblock}{Lei de Demeter}
Um objeto deve falar apenas com seus vizinhos imediatos, não com estranhos.
\end{alertblock}
\end{frame}

\begin{frame}
\frametitle{Encapsulamento em Prática}

\begin{columns}
\begin{column}{0.5\textwidth}
\begin{block}{\textcolor{cleanred}{\faTimes} Violação do Encapsulamento}
\lstinputlisting[language=Java]{examples/bad_encapsulation.java}
\end{block}
\end{column}

\begin{column}{0.5\textwidth}
\begin{block}{\textcolor{cleangreen}{\faCheck} Encapsulamento Adequado}
\lstinputlisting[language=Java]{examples/good_encapsulation.java}
\end{block}
\end{column}
\end{columns}

\vspace{0.5cm}
\begin{itemize}
    \item \textbf{Dados privados:} Protegem o estado interno
    \item \textbf{Métodos públicos:} Interface controlada
    \item \textbf{Comportamento:} Lógica encapsulada no objeto
\end{itemize}
\end{frame}

\section{Conclusão}

\begin{frame}
\frametitle{Resumo dos Princípios}

\begin{center}
\begin{tikzpicture}[node distance=2cm]
    \node[rectangle, draw, fill=cleanblue!20, text width=2cm, align=center] (nomes) {Nomes\\Significativos};
    \node[rectangle, draw, fill=cleangreen!20, text width=2cm, align=center, right=of nomes] (funcoes) {Funções\\Pequenas};
    \node[rectangle, draw, fill=yellow!20, text width=2cm, align=center, right=of funcoes] (comentarios) {Comentários\\Úteis};
    
    \node[rectangle, draw, fill=orange!20, text width=2cm, align=center, below=1cm of nomes] (formatacao) {Formatação\\Consistente};
    \node[rectangle, draw, fill=purple!20, text width=2cm, align=center, right=of formatacao] (objetos) {Objetos\\Bem Definidos};
    
    % Setas conectoras
    \draw[->, thick] (nomes) -- (funcoes);
    \draw[->, thick] (funcoes) -- (comentarios);
    \draw[->, thick] (nomes) -- (formatacao);
    \draw[->, thick] (formatacao) -- (objetos);
    \draw[->, thick] (funcoes) -- (objetos);
\end{tikzpicture}
\end{center}

\vspace{0.5cm}
\begin{alertblock}{Lembre-se}
\centering
Código limpo não é escrito de uma vez. É refinado continuamente.
\end{alertblock}
\end{frame}

\begin{frame}
\frametitle{Próximos Passos}

\begin{enumerate}
    \item \textbf{Pratique:} Aplique esses princípios no seu código diário
    \item \textbf{Refatore:} Melhore código existente gradualmente
    \item \textbf{Code Review:} Use esses critérios para avaliar código
    \item \textbf{Ferramentas:} Utilize analisadores estáticos (próxima aula)
\end{enumerate}

\vspace{0.5cm}
\begin{block}{Para a Próxima Aula}
\begin{itemize}
    \item Leitura: Capítulo sobre Code Smells
    \item Exercício: Identificar code smells em projeto pessoal
    \item Ferramenta: Instalar SonarQube/PMD
\end{itemize}
\end{block}

\vspace{0.5cm}
\begin{center}
\textbf{Dúvidas?}
\end{center}
\end{frame}

\begin{frame}
\frametitle{Referências}

\begin{itemize}
    \item Martin, Robert C. \textbf{Clean Code: A Handbook of Agile Software Craftsmanship}. Prentice Hall, 2008.
    \item Fowler, Martin. \textbf{Refactoring: Improving the Design of Existing Code}. Addison-Wesley, 2019.
    \item Catálogo de Code Smells: \url{https://luzkan.github.io/smells/}
    \item Repositório do curso: \url{https://github.com/fmarquesfilho/bpp-2025-2}
\end{itemize}

\vspace{1cm}
\begin{center}
\Large
Obrigado pela atenção!\\
\vspace{0.5cm}
\normalsize
\texttt{fernando.marques@ufrn.br}
\end{center}
\end{frame}


\end{document}