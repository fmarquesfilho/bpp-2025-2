\documentclass[10pt]{beamer}
\usepackage[utf8]{inputenc}
\usepackage[T1]{fontenc}
\usepackage[brazilian]{babel}
\usepackage{tikz}
\usetikzlibrary{shapes.geometric, arrows, positioning, matrix}
\usepackage{amssymb}
\usepackage{xcolor}
\usepackage{listings}
\usetheme{Madrid}
\usepackage{pifont}

\title{Boas Práticas de Programação (2025.2)}
\subtitle{Projeto Integrado: MVP + Código Limpo + Refatoração}
\author{Prof. Fernando Figueira}
\institute{DIMAp - UFRN}
\date{Agosto de 2025}

% Definindo estilos personalizados
\tikzset{
    process/.style = {rectangle, rounded corners, minimum width=3cm, minimum height=1cm, text centered, draw=black, fill=blue!20},
    priority1/.style = {rectangle, rounded corners, minimum width=8cm, minimum height=0.8cm, text centered, draw=black, fill=green!30},
    priority2/.style = {rectangle, rounded corners, minimum width=8cm, minimum height=0.8cm, text centered, draw=black, fill=yellow!30},
    priority3/.style = {rectangle, rounded corners, minimum width=8cm, minimum height=0.8cm, text centered, draw=black, fill=orange!30},
    arrow/.style = {thick,->,>=stealth}
}

% Configuração para código
\lstset{
    basicstyle=\tiny\ttfamily,
    breaklines=true,
    frame=single,
    numbers=left,
    numberstyle=\tiny,
    showstringspaces=false,
    backgroundcolor=\color{gray!10}
}

\begin{document}

\frame{\titlepage}

% Slide 1 - Overview do Projeto Integrado
\begin{frame}{Projeto Integrado: MVP + Qualidade de Código}
\begin{columns}[c]
\column{0.5\textwidth}
\textbf{Objetivos do Projeto:}
\begin{itemize}
    \item \textbf{Entregar Valor}: MVP funcional
    \item \textbf{Código Limpo}: Nomenclatura e estrutura
    \item \textbf{Identificar Problemas}: Code smells
    \item \textbf{Melhorar Continuamente}: Refatoração
\end{itemize}

\column{0.5\textwidth}
\centering
\begin{tikzpicture}[scale=0.6]
    \draw[fill=blue!20] (0,0) circle (2cm) node {\textbf{Seu Projeto}};
    \draw[fill=green!20] (-1.5,1.5) circle (0.8cm) node[font=\tiny] {MVP};
    \draw[fill=yellow!20] (1.5,1.5) circle (0.8cm) node[font=\tiny] {Clean Code};
    \draw[fill=orange!20] (1.5,-1.5) circle (0.8cm) node[font=\tiny] {Refactoring};
    \draw[fill=red!20] (-1.5,-1.5) circle (0.8cm) node[font=\tiny] {Quality};
\end{tikzpicture}
\end{columns}

\vspace{0.3cm}
\begin{alertblock}{Nova Abordagem}
Seu projeto será avaliado tanto pelo \textbf{valor entregue} quanto pela \textbf{qualidade do código}!
\end{alertblock}
\end{frame}

% Slide 2 - Visão do Produto (mantido)
\begin{frame}{Visão do Produto: Template Prático}
\begin{block}{Template da Visão}
\textbf{Para} [usuários-alvo] \\
\textbf{Que} [problema/necessidade] \\
\textbf{O} [nome do produto] \textbf{é um} [categoria] \\
\textbf{Que} [benefício principal] \\
\textbf{Diferente de} [alternativa existente] \\
\textbf{Nosso produto} [diferencial único]
\end{block}

\vspace{0.3cm}
\begin{exampleblock}{Checklist da Visão}
\small
\checkmark\ Define usuário-alvo específico \\
\checkmark\ Identifica problema concreto \\
\checkmark\ Explicita valor único \\
\checkmark\ É inspiradora mas realista \\
\checkmark\ \textbf{Permite aplicar boas práticas de código}
\end{exampleblock}
\end{frame}

% Slide 3 - MVP com Critérios de Qualidade
\begin{frame}{Framework para MVP + Qualidade}
\begin{block}{1. Problema Core}
Qual o \textbf{principal} problema que seu produto resolve?
\end{block}

\begin{block}{2. Hipótese de Valor}
"Acreditamos que [usuários] vão [comportamento] porque [benefício]"
\end{block}

\begin{block}{3. Critérios de Qualidade}
Como garantir que o código seja limpo e de fácil manutenção?
\end{block}

\vspace{0.3cm}
\begin{exampleblock}{Exemplo - Sistema de Biblioteca}
\textbf{Problema}: Estudantes perdem tempo procurando livros \\
\textbf{MVP}: Busca + Status + Empréstimo simples \\
\textbf{Qualidade}: Funções < 20 linhas, nomes descritivos
\end{exampleblock}
\end{frame}

% Slide 4 - Código Limpo no MVP
\begin{frame}{Princípios de Código Limpo no MVP}
\begin{columns}[c]
\column{0.5\textwidth}
\textbf{Nomenclatura:}
\begin{itemize}
    \item Nomes intencionais
    \item Evitar abreviações
    \item Usar termos do domínio
    \item Pronunciáveis
\end{itemize}

\textbf{Funções:}
\begin{itemize}
    \item Pequenas (< 20 linhas)
    \item Uma responsabilidade
    \item Poucos parâmetros
    \item Nome descritivo
\end{itemize}

\column{0.5\textwidth}
\textbf{Formatação:}
\begin{itemize}
    \item Indentação consistente
    \item Espaçamento vertical
    \item Agrupamento lógico
    \item Linhas < 120 chars
\end{itemize}

\textbf{Comentários:}
\begin{itemize}
    \item Apenas quando necessário
    \item Explicam "por quê"
    \item Mantidos atualizados
    \item Não redundantes
\end{itemize}
\end{columns}
\end{frame}

% Slide 6 - Code Smells Prioritários
\begin{frame}{Code Smells: O que Procurar}
\begin{columns}[t]
\column{0.48\textwidth}
\textbf{Nível Método/Função:}
\begin{itemize}
    \item \textbf{Long Method}: > 20-30 linhas
    \item \textbf{Long Parameter List}: > 3-4 parâmetros  
    \item \textbf{Duplicate Code}: Repetição
    \item \textbf{Dead Code}: Código não usado
\end{itemize}

\textbf{Nível Classe:}
\begin{itemize}
    \item \textbf{Large Class}: Muitas responsabilidades
    \item \textbf{Data Class}: Só dados, sem comportamento
    \item \textbf{God Class}: Classe que faz tudo
\end{itemize}

\column{0.48\textwidth}
\textbf{Nível Estrutural:}
\begin{itemize}
    \item \textbf{Feature Envy}: Método interessado em outra classe
    \item \textbf{Inappropriate Intimacy}: Classes muito acopladas
    \item \textbf{Poor Naming}: Nomes ambíguos
\end{itemize}

\vspace{0.3cm}
\begin{alertblock}{Objetivo}
Identificar pelo menos \textbf{3 code smells} diferentes no seu código!
\end{alertblock}
\end{columns}
\end{frame}

% Slide 7 - Técnicas de Refatoração
\begin{frame}{Técnicas de Refatoração Essenciais}
\begin{columns}[c]
\column{0.5\textwidth}
\textbf{Extract Method:}
\begin{itemize}
    \item Transformar trecho em método
    \item Reduzir tamanho de funções
    \item Melhorar legibilidade
\end{itemize}

\textbf{Rename Variable/Method:}
\begin{itemize}
    \item Nomes mais descritivos
    \item Eliminar ambiguidade
    \item Usar vocabulário do domínio
\end{itemize}

\column{0.5\textwidth}
\textbf{Introduce Parameter Object:}
\begin{itemize}
    \item Agrupar parâmetros relacionados
    \item Reduzir lista de parâmetros
    \item Criar classes de dados
\end{itemize}

\textbf{Remove Duplicate Code:}
\begin{itemize}
    \item Extrair código comum
    \item Criar funções reutilizáveis
    \item Manter consistência
\end{itemize}
\end{columns}

\vspace{0.3cm}
\begin{exampleblock}{Meta}
Aplicar pelo menos \textbf{3 refatorações} documentadas no seu projeto!
\end{exampleblock}
\end{frame}

% Slide 8 - MoSCoW + Critérios de Qualidade
\begin{frame}{MoSCoW + Critérios de Qualidade}
\centering
\begin{tikzpicture}[scale=0.9]
    \node[priority1, text width=7cm] at (0,0) {\textbf{Must Have} - MVP + Código Limpo};
    \node[priority2, text width=7cm] at (0,-1) {\textbf{Should Have} - Refatorações + SOLID};
    \node[priority3, text width=7cm] at (0,-2) {\textbf{Could Have} - Testes + Métricas};
    \node[draw, fill=red!20, rounded corners, minimum width=7cm, minimum height=0.8cm, text centered] at (0,-3) {\textbf{Won't Have} - Features complexas desnecessárias};
\end{tikzpicture}

\vspace{0.5cm}
\textbf{Lembre-se}: Melhor um MVP simples com código de qualidade que features demais mal implementadas!
\end{frame}

% Slide 9 - Backlog com Critérios de Qualidade
\begin{frame}{Backlog: Funcionalidade + Qualidade}
\scriptsize
\begin{tabular}{|c|p{4cm}|p{4cm}|c|}
\hline
\textbf{Pri} & \textbf{User Story} & \textbf{Critérios de Qualidade} & \textbf{Est} \\
\hline
P1 & Como estudante, quero cadastrar tarefa & - Função cadastro < 20 linhas\newline - Nomenclatura descritiva\newline - Validação de entrada & 4h \\
\hline
P1 & Como estudante, quero listar tarefas & - Separar lógica/apresentação\newline - Função reutilizável\newline - Tratamento de lista vazia & 3h \\
\hline
P2 & \textbf{Refatorar código duplicado} & - Identificar duplicação\newline - Extrair função comum\newline - Documentar refatoração & 2h \\
\hline
P2 & \textbf{Melhorar nomenclatura} & - Revisar nomes ambíguos\newline - Aplicar convenções\newline - Atualizar documentação & 1h \\
\hline
\end{tabular}

\vspace{0.3cm}
\textbf{Novidade}: Itens de \textbf{qualidade} também entram no backlog!
\end{frame}

% Slide 10 - Cronograma Integrado
\begin{frame}{Cronograma Integrado com Conteúdo}
\scriptsize
\begin{tikzpicture}[scale=0.8]
    % Semanas
    \draw[fill=blue!10] (0,3.5) rectangle (2.2,4.5) node[midway, font=\tiny] {Sem 1: Setup};
    \draw[fill=green!20] (2.4,3.5) rectangle (4.6,4.5) node[midway, font=\tiny] {Sem 2: Clean Code};
    \draw[fill=yellow!20] (4.8,3.5) rectangle (7,4.5) node[midway, font=\tiny] {Sem 3: Code Smells};
    \draw[fill=orange!20] (7.2,3.5) rectangle (9.4,4.5) node[midway, font=\tiny] {Sem 4: SOLID};
    \draw[fill=red!20] (9.6,3.5) rectangle (11.8,4.5) node[midway, font=\tiny] {Sem 5: Refatoração};
    
    % Atividades detalhadas
    \node[below, font=\tiny, text width=2cm, align=center] at (1.1,3.5) {
        • Visão produto\\
        • MVP inicial\\
        • Ambiente dev
    };
    \node[below, font=\tiny, text width=2cm, align=center] at (3.5,3.5) {
        • Nomenclatura\\
        • Formatação\\
        • Funções pequenas
    };
    \node[below, font=\tiny, text width=2cm, align=center] at (5.9,3.5) {
        • Identificar smells\\
        • Ferramentas\\
        • Catalogar problemas
    };
    \node[below, font=\tiny, text width=2cm, align=center] at (8.3,3.5) {
        • Responsabilidades\\
        • Reorganizar classes\\
        • Reduzir acoplamento
    };
    \node[below, font=\tiny, text width=2cm, align=center] at (10.7,3.5) {
        • Aplicar técnicas\\
        • Documentar\\
        • Finalizar entrega
    };
\end{tikzpicture}
\normalsize

\vspace{0.3cm}
\textbf{Cada semana}: Desenvolvimento + aplicação dos conceitos da aula
\end{frame}

% Slide 11 - Definition of Done Progressivo
\begin{frame}{Definition of Done com Qualidade}
\begin{columns}[t]
\column{0.32\textwidth}
\textbf{Sprint 1}
\begin{itemize}
    \item Funcionalidade OK
    \item Código compila
    \item \textbf{Nomes descritivos}
    \item \textbf{Formatação consistente}
    \item \textbf{Funções < 20 linhas}
\end{itemize}

\column{0.32\textwidth}
\textbf{Sprint 2}
\begin{itemize}
    \item Tudo do Sprint 1 +
    \item \textbf{Code smells catalogados}
    \item \textbf{2+ smells corrigidos}
    \item Tratamento de erros
    \item Documentação inicial
\end{itemize}

\column{0.32\textwidth}
\textbf{Sprint 3}
\begin{itemize}
    \item Tudo do Sprint 2 +
    \item \textbf{SOLID aplicado}
    \item \textbf{3+ refatorações}
    \item \textbf{Relatório qualidade}
    \item Code review próprio
\end{itemize}
\end{columns}

\vspace{0.5cm}
\begin{alertblock}{Evolução}
Definition of Done evolui incorporando conceitos das aulas!
\end{alertblock}
\end{frame}

% Slide 12 - Ferramentas de Análise
\begin{frame}{Ferramentas para Análise de Qualidade}
\begin{columns}[c]
\column{0.5\textwidth}
\textbf{Python:}
\begin{itemize}
    \item \textbf{pylint}: Análise estática completa
    \item \textbf{flake8}: Style guide enforcement
    \item \textbf{black}: Formatação automática
    \item \textbf{radon}: Métricas de complexidade
\end{itemize}

\textbf{Java:}
\begin{itemize}
    \item \textbf{Checkstyle}: Convenções de código
    \item \textbf{PMD}: Code smells detector
    \item \textbf{SpotBugs}: Bug pattern detection
\end{itemize}

\column{0.5\textwidth}
\textbf{JavaScript:}
\begin{itemize}
    \item \textbf{ESLint}: Linting e qualidade
    \item \textbf{Prettier}: Formatação consistente
    \item \textbf{SonarJS}: Análise de qualidade
\end{itemize}

\textbf{Multiplataforma:}
\begin{itemize}
    \item \textbf{SonarLint}: Plugin IDE
    \item \textbf{SonarCloud}: Análise online
    \item \textbf{CodeClimate}: Métricas contínuas
\end{itemize}
\end{columns}

\vspace{0.3cm}
\begin{block}{Recomendação}
Use pelo menos \textbf{1 ferramenta} para identificar code smells automaticamente!
\end{block}
\end{frame}

% Slide 13 - Métricas de Qualidade
\begin{frame}{Métricas de Qualidade para Acompanhar}
\begin{columns}[c]
\column{0.5\textwidth}
\textbf{Métricas Básicas:}
\begin{itemize}
    \item \textbf{Linhas por Método}: < 20-30
    \item \textbf{Parâmetros por Função}: < 4
    \item \textbf{Complexidade Ciclomática}: < 10
    \item \textbf{Duplicação}: < 5%
\end{itemize}

\column{0.5\textwidth}
\textbf{Indicadores de Qualidade:}
\begin{itemize}
    \item \textbf{Nomes descritivos}: 90%+
    \item \textbf{Comentários úteis}: Não redundantes
    \item \textbf{Code smells}: Identificados e catalogados
    \item \textbf{Refatorações}: Documentadas
\end{itemize}
\end{columns}

\vspace{0.5cm}
\begin{exampleblock}{Meta do Projeto}
\textbf{3+ code smells} identificados e \textbf{3+ refatorações} aplicadas com documentação completa
\end{exampleblock}
\end{frame}

% Slide 14 - Estrutura de Projeto
\begin{frame}{Estrutura de Projeto + Qualidade}
\begin{columns}[c]
\column{0.5\textwidth}
\begin{block}{Estrutura Recomendada}
\texttt{projeto/}\\
\texttt{|-- src/} \hfill \textit{Código-fonte}\\
\texttt{|-- tests/} \hfill \textit{Testes}\\
\texttt{|-- docs/} \hfill \textit{Documentação}\\
\texttt{|-- refactoring/} \hfill \textit{\textbf{Análises e refatorações}}\\
\texttt{|-- tools/} \hfill \textit{Scripts de análise}\\
\texttt{|-- README.md} \hfill \textit{Visão geral + qualidade}
\end{block}

\column{0.5\textwidth}
\textbf{Pasta /refactoring/:}
\begin{itemize}
    \item code-smells-identified.md
    \item refactoring-log.md
    \item before-after-examples/
    \item quality-metrics.md
\end{itemize}

\textbf{README.md deve incluir:}
\begin{itemize}
    \item Como executar análises
    \item Convenções de código
    \item Métricas atuais
\end{itemize}
\end{columns}
\end{frame}

% Slide 15 - Exemplo de Documentação de Refatoração
\begin{frame}[fragile]{Exemplo: Documentação de Refatoração}
\textbf{Template para refactoring-log.md:}

\begin{block}{Refatoração 1: Extract Method}
\textbf{Code Smell}: Long Method em \texttt{process\_data()} - 45 linhas

\textbf{Técnica Aplicada}: Extract Method

\textbf{Justificativa}: Método fazia validação + processamento + persistência

\textbf{Resultado}: 
\begin{itemize}
    \item \texttt{validate\_input()}: 8 linhas
    \item \texttt{process\_business\_logic()}: 12 linhas  
    \item \texttt{save\_to\_database()}: 6 linhas
\end{itemize}

\textbf{Impacto}: Melhor testabilidade e legibilidade
\end{block}

\vspace{0.2cm}
\textbf{Documente}: Cada refatoração com antes/depois e justificativa!
\end{frame}

% Slide 16 - Critérios de Avaliação
\begin{frame}{Critérios de Avaliação - Unidade 1}
\begin{columns}[t]
\column{0.5\textwidth}
\textbf{Qualidade do Código (30\%)}
\begin{itemize}
    \item Aplicação de código limpo
    \item Nomenclatura e estrutura
    \item Formatação consistente
\end{itemize}

\textbf{Code Smells (20\%)}
\begin{itemize}
    \item Identificação correta
    \item Uso de ferramentas
    \item Catalogação detalhada
\end{itemize}

\column{0.5\textwidth}
\textbf{Refatorações (20\%)}
\begin{itemize}
    \item Técnicas bem aplicadas
    \item Documentação completa
    \item Melhoria efetiva
\end{itemize}

\textbf{MVP e Visão (15\% cada)}
\begin{itemize}
    \item Funcionalidade e valor
    \item Planejamento coerente
\end{itemize}
\end{columns}
\end{frame}

% Slide 17 - Documentação Obrigatória
\begin{frame}{Documentação Obrigatória Expandida}
\textbf{Documentos de Entrega:}
\begin{itemize}
    \item[\checkmark] Visão do Produto (PDF, 2-3 páginas)
    \item[\checkmark] Product Backlog (PDF)
    \item[\checkmark] Relatório de Qualidade de Código (PDF, 3-4 páginas)
    \item[\checkmark] Link para o repositório com o código-fonte completo
    \item[\checkmark] Vídeo de apresentação (8-10 minutos)
\end{itemize}

\vspace{0.3cm}
\textbf{Estrutura do Relatório de Qualidade:}
\begin{enumerate}
    \item Aplicação de Código Limpo (exemplos)
    \item Code Smells Identificados (tabela + análise)  
    \item Refatorações Realizadas (antes/depois)
    \item Ferramentas Utilizadas (prints + métricas)
    \item Próximos Passos (melhorias planejadas)
\end{enumerate}
\end{frame}

% Slide 18 - Estrutura do Vídeo Atualizada
\begin{frame}{Estrutura do Vídeo (8-10 min)}
\textbf{Roteiro Sugerido:}

\begin{columns}[t]
\column{0.5\textwidth}
\textbf{Min 1-2}: Problema e Visão
\begin{itemize}
    \item Apresentação do problema
    \item Solução proposta
    \item MVP definido
\end{itemize}

\textbf{Min 3-4}: Demo do MVP
\begin{itemize}
    \item Funcionalidades principais
    \item Navegação pelo sistema
    \item Valor entregue
\end{itemize}

\column{0.5\textwidth}
\textbf{Min 5-6}: Qualidade do Código
\begin{itemize}
    \item Exemplos de código limpo
    \item Estrutura organizada
    \item Boas práticas aplicadas
\end{itemize}

\textbf{Min 7-8}: Code Smells e Refatoração
\begin{itemize}
    \item Problemas identificados
    \item Técnicas aplicadas
    \item Melhorias alcançadas
\end{itemize}

\textbf{Min 9-10}: Conclusões
\begin{itemize}
    \item Lições aprendidas
    \item Próximos passos
\end{itemize}
\end{columns}
\end{frame}

% Slide 19 - Sinais de Bom Projeto Atualizado
\begin{frame}{Sinais de um Bom Projeto}
\begin{columns}[c]
\column{0.5\textwidth}
\textbf{Sinais Positivos:}
\begin{itemize}
    \item \checkmark\ MVP funcional e valioso
    \item \checkmark\ \textbf{Código é legível}
    \item \checkmark\ \textbf{Funções pequenas e focadas}
    \item \checkmark\ \textbf{Nomes são descritivos}
    \item \checkmark\ Code smells identificados honestamente
    \item \checkmark\ Refatorações bem justificadas
\end{itemize}

\column{0.5\textwidth}
\textbf{Sinais de Atenção:}
\begin{itemize}
    \item \ding{55}\ Code smells ignorados
    \item \ding{55}\ Funções muito longas
    \item \ding{55}\ Nomes ambíguos (a, tmp, data)
    \item \ding{55}\ Código duplicado sem correção
    \item \ding{55}\ Refatorações superficiais
    \item \ding{55}\ Foco só nas features
\end{itemize}
\end{columns}

\vspace{0.5cm}
\begin{alertblock}{Lembre-se}
Código de qualidade hoje evita dor de cabeça amanhã!
\end{alertblock}
\end{frame}

% Slide 20 - Erros Comuns Atualizados
\begin{frame}{Erros Comuns a Evitar}
\begin{enumerate}
    \item \textbf{Ignorar qualidade de código}
    \begin{itemize}
        \item[\ding{55}] "Vou limpar depois"
        \item[\checkmark] "Código limpo desde o início"
    \end{itemize}
    
    \item \textbf{Code smells "falsos"}
    \begin{itemize}
        \item[\ding{55}] Inventar problemas inexistentes
        \item[\checkmark] Usar ferramentas para detecção real
    \end{itemize}
    
    \item \textbf{Refatorações cosméticas}
    \begin{itemize}
        \item[\ding{55}] Apenas renomear variáveis
        \item[\checkmark] Melhorias estruturais significativas
    \end{itemize}
    
    \item \textbf{Documentação insuficiente}
    \begin{itemize}
        \item[\ding{55}] "O código se explica"
        \item[\checkmark] Justificar todas as decisões
    \end{itemize}
\end{enumerate}
\end{frame}

% Slide 21 - Checklist Expandido
\begin{frame}{Checklist Final Expandido}
\begin{columns}[t]
\column{0.5\textwidth}
\textbf{Código e Qualidade:}
\begin{itemize}
    \item[\checkmark] Nomes são descritivos
    \item[\checkmark] Funções < 20 linhas
    \item[\checkmark] Código formatado consistente
    \item[\checkmark] \textbf{3+ code smells identificados}
    \item[\checkmark] \textbf{3+ refatorações aplicadas}
    \item[\checkmark] Ferramentas de análise usadas
\end{itemize}

\column{0.5\textwidth}
\textbf{Documentação:}
\begin{itemize}
    \item[\checkmark] Relatório de qualidade completo
    \item[\checkmark] Refatorações bem documentadas
    \item[\checkmark] Link para repositório do projeto
    \item[\checkmark] Before/after examples
    \item[\checkmark] Vídeo mostra código e qualidade
    \item[\checkmark] README com padrões de qualidade
\end{itemize}
\end{columns}

\vspace{0.5cm}
\begin{block}{Autoavaliação Final}
"Outro desenvolvedor conseguiria entender e manter meu código facilmente?"
\end{block}
\end{frame}

% Slide 22 - Recursos e Apoio
\begin{frame}{Recursos de Apoio Disponíveis}
\textbf{Catálogos de Referência:}
\begin{itemize}
    \item \textbf{Code Smells}: \url{https://luzkan.github.io/smells/}
    \item \textbf{Refactoring}: Martin Fowler's catalog
    \item \textbf{Clean Code}: Robert C. Martin principles
\end{itemize}

\vspace{0.3cm}
\textbf{Ferramentas Online:}
\begin{itemize}
    \item \textbf{SonarCloud}: Análise gratuita projetos públicos
    \item \textbf{CodeClimate}: Métricas de qualidade
    \item \textbf{Better Code Hub}: Compliance com guidelines
\end{itemize}

\vspace{0.3cm}
\textbf{Apoio da Disciplina:}
\begin{itemize}
    \item \textbf{Atendimento}: Segundas 14h-16h (online)
    \item \textbf{Discord}: \url{https://discord.gg/bbMFJBQRT8}
    \item \textbf{GitHub}: \url{https://github.com/fmarquesfilho/bpp-2025-2}
\end{itemize}
\end{frame}

% Slide Final
\begin{frame}{Sucesso no Projeto!}
\centering

\vspace{0.5cm}
\textbf{Dúvidas?} fernando@dimap.ufrn.br
\end{frame}

\end{document}
