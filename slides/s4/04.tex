% aula_04.tex - Code Smells + SOLID
\documentclass[aspectratio=169]{beamer}
\usepackage[utf8]{inputenc}
\usepackage[T1]{fontenc}
\usepackage[brazil]{babel}
\usepackage{listings}
\usepackage{xcolor}
\usepackage{tikz}
\usepackage{fontawesome5}
\usepackage{graphicx}
\usepackage{url}
\usepackage{hyperref}
\usetikzlibrary{shapes.geometric, arrows, positioning, shadows}

% Tema moderno
\usetheme{metropolis}
\usecolortheme{default}

% Cores personalizadas
\definecolor{cleanblue}{RGB}{52, 152, 219}
\definecolor{cleangreen}{RGB}{46, 204, 113}
\definecolor{cleanred}{RGB}{231, 76, 60}
\definecolor{cleangray}{RGB}{149, 165, 166}
\definecolor{darkgray}{RGB}{44, 62, 80}

% Configuração do código Python
\lstdefinestyle{pythonstyle}{
    language=Python,
    backgroundcolor=\color{gray!10},
    commentstyle=\color{cleangreen},
    keywordstyle=\color{cleanblue}\bfseries,
    numberstyle=\tiny\color{cleangray},
    stringstyle=\color{cleanred},
    basicstyle=\scriptsize\ttfamily,
    breakatwhitespace=false,
    breaklines=true,
    captionpos=b,
    keepspaces=true,
    numbers=left,
    numbersep=3pt,
    showspaces=false,
    showstringspaces=false,
    showtabs=false,
    tabsize=2,
    frame=single,
    rulecolor=\color{cleangray!30},
    xleftmargin=10pt,
    xrightmargin=10pt,
    inputencoding=utf8,  % ← ADICIONE ESTA LINHA
    extendedchars=true,         % ← ADICIONE ESTA LINHA
    literate=                   % ← ADICIONE ESTAS LINHAS
      {á}{{\'a}}1 {é}{{\'e}}1 {í}{{\'i}}1 {ó}{{\'o}}1 {ú}{{\'u}}1
      {Á}{{\'A}}1 {É}{{\'E}}1 {Í}{{\'I}}1 {Ó}{{\'O}}1 {Ú}{{\'U}}1
      {à}{{\`a}}1 {è}{{\`e}}1 {ì}{{\`i}}1 {ò}{{\`o}}1 {ù}{{\`u}}1
      {À}{{\`A}}1 {È}{{\'E}}1 {Ì}{{\`I}}1 {Ò}{{\`O}}1 {Ù}{{\`U}}1
      {ä}{{\"a}}1 {ë}{{\"e}}1 {ï}{{\"i}}1 {ö}{{\"o}}1 {ü}{{\"u}}1
      {Ä}{{\"A}}1 {Ë}{{\"E}}1 {Ï}{{\"I}}1 {Ö}{{\"O}}1 {Ü}{{\"U}}1
      {â}{{\^a}}1 {ê}{{\^e}}1 {î}{{\^i}}1 {ô}{{\^o}}1 {û}{{\^u}}1
      {Â}{{\^A}}1 {Ê}{{\^E}}1 {Î}{{\^I}}1 {Ô}{{\^O}}1 {Û}{{\^U}}1
      {ã}{{\~a}}1 {ẽ}{{\~e}}1 {ĩ}{{\~i}}1 {õ}{{\~o}}1 {ũ}{{\~u}}1
      {Ã}{{\~A}}1 {Ẽ}{{\~E}}1 {Ĩ}{{\~I}}1 {Õ}{{\~O}}1 {Ũ}{{\~U}}1
      {ç}{{\c{c}}}1 {Ç}{{\c{C}}}1
      {ñ}{{\~n}}1 {Ñ}{{\~N}}1
}


\lstset{style=pythonstyle}

% Informações do documento
\title{Code Smells + Princípios SOLID}
\subtitle{Boas Práticas de Programação - BPP 2025.2}
\author{Prof. Fernando Marques Filho}
\institute{Universidade Federal do Rio Grande do Norte}
\date{12 de Setembro de 2025}

\begin{document}

\frame{\titlepage}

\begin{frame}
\frametitle{Agenda}
\tableofcontents
\end{frame}

\section{Introdução a Code Smells}

\begin{frame}
\frametitle{O que são Code Smells?}
\begin{center}
\begin{tikzpicture}
    \node[draw, rectangle, fill=cleanred!20, minimum width=8cm, minimum height=2.0cm, rounded corners=10pt] (smell) {
        \begin{minipage}{6.0cm}
        \centering
        \textbf{Sinais de problemas no código} \\
        \footnotesize Indicam possíveis violações de boas práticas \\
        \vspace{0.2cm}
        \textit{Não são bugs, mas podem levar a eles}
        \end{minipage}
    };
\end{tikzpicture}
\end{center}

\begin{block}{Definição}
\vspace{0.1cm}
\footnotesize
Code smells são características superficiais no código-fonte que podem indicar problemas mais profundos no design, estrutura ou implementação.
\end{block}


\begin{exampleblock}{\small{Exemplos Comuns}}
\footnotesize
\begin{itemize}
    \item \textbf{Long Method} - Métodos muito longos
    \item \textbf{Duplicate Code} - Código duplicado
    \item \textbf{Large Class} - Classes muito grandes
    \item \textbf{God Class} - Classe que faz tudo
\end{itemize}
\end{exampleblock}
\end{frame}

\begin{frame}
\frametitle{Origem e Importância}
\begin{columns}
\begin{column}{0.6\textwidth}
\footnotesize
\textbf{Termo criado por:} Martin Fowler (1999) \\
\textbf{Inspiração:} Kent Beck \\
\textbf{Livro:} "Refactoring: Improving the Design of Existing Code"

\vspace{0.5cm}
\textbf{Por que são importantes?}
\begin{itemize}
    \item Facilitam identificação de problemas
    \item Guiam decisões de refatoração
    \item Melhoram manutenibilidade
    \item Reduzem débito técnico
\end{itemize}
\end{column}

\begin{column}{0.4\textwidth}
\begin{center}
\begin{tikzpicture}[scale=1.1]
    \draw[fill=yellow!20, draw=orange, thick] (0,0) rectangle (3,3.2);
    \node at (1.5,2.5) {\faExclamationTriangle};
    \node[text width=2.8cm, align=center, font=\scriptsize] at (1.5,1.8) {\textbf{Code Smell}};
    \node[text width=2.8cm, align=center, font=\tiny] at (1.5,1.1) {Sinal de que algo pode estar errado no design};
    \node[text width=2.8cm, align=center, font=\tiny] at (1.5,0.5) {Não é um bug, mas pode levar a problemas};
\end{tikzpicture}
\end{center}
\end{column}
\end{columns}
\end{frame}

\begin{frame}
\frametitle{Catálogo de Code Smells}
\begin{center}
\textbf{Recurso Oficial:} \\
\vspace{0.3cm}
\Large \url{https://luzkan.github.io/smells/}
\end{center}

\vspace{0.5cm}
\begin{columns}
\begin{column}{0.5\textwidth}
\begin{block}{Categorias Principais}
\footnotesize
\begin{itemize}
    \item \textbf{Bloaters} - Código inchado
    \item \textbf{Object-Orientation Abusers} - Abuso de OO
    \item \textbf{Change Preventers} - Impedimentos de mudança
    \item \textbf{Dispensables} - Código desnecessário
    \item \textbf{Couplers} - Acoplamento excessivo
\end{itemize}
\end{block}
\end{column}

\begin{column}{0.5\textwidth}
\begin{alertblock}{Projeto BPP 2025.2}
\footnotesize
Consulte também os exemplos no repositório: \\
\vspace{0.2cm}
\href{https://github.com/fmarquesfilho/bpp-2025-2}{Clique aqui para abrir}
\end{alertblock}
\end{column}
\end{columns}
\end{frame}

\begin{frame}[fragile]
\frametitle{Exemplo: Long Method}
\begin{block}{\textcolor{cleanred}{\faTimes} Violação}
\tiny
\begin{lstlisting}
def process_user_registration(name, email, password, age):
    # Validacao (15+ linhas no total)
    if not name or len(name) < 2:
        raise ValueError("Nome invalido")
    if not email or "@" not in email:
        raise ValueError("Email invalido") 
    if not password or len(password) < 8:
        raise ValueError("Senha muito fraca")
    if age < 18 or age > 120:
        raise ValueError("Idade invalida")

    # ... mais codigo aqui
    return True
\end{lstlisting}
\end{block}

\vspace{0.1cm}
\footnotesize
\faGithub \, \href{https://github.com/fmarquesfilho/bpp-2025-2/}{Clique aqui} para ver exemplo completo.
\end{frame}

\begin{frame}[fragile]
\frametitle{Exemplo: Long Method Refatorado}
\begin{block}{\textcolor{cleangreen}{\faCheck} Solução}
\scriptsize
\begin{lstlisting}
def process_user_registration(user_data):
    """Processa registro aplicando validacao, formatacao e persistencia."""
    validate_user_input(user_data)
    formatted_data = format_user_data(user_data)
    save_user(formatted_data)
    send_welcome_email(formatted_data.email)
    return True

def validate_user_input(user_data):
    """Valida todos os dados de entrada do usuario."""
    if not is_valid_name(user_data.name):
        raise ValueError("Nome invalido")
    # ... outras validacoes
\end{lstlisting}
\end{block}

\vspace{0.1cm}
\footnotesize
\faGithub \, \href{https://github.com/fmarquesfilho/bpp-2025-2/blob/main/src/s4/problema1.py}{Clique aqui} para ver exemplo completo.
\end{frame}

\section{Ferramentas de Detecção}

\begin{frame}
\frametitle{Ferramentas Python para Análise}
\begin{columns}
\begin{column}{0.5\textwidth}
\begin{block}{pylint}
\footnotesize
Análise estática completa\\
\texttt{pip install pylint}\\
\texttt{pylint meu\_projeto/}
\end{block}

\begin{block}{flake8}
\footnotesize
Estilo e complexidade\\
\texttt{pip install flake8}\\
\texttt{flake8 --max-complexity=10}
\end{block}
\end{column}

\begin{column}{0.5\textwidth}
\begin{block}{radon}
\footnotesize
Métricas de complexidade\\
\texttt{pip install radon}\\
\texttt{radon cc . -a}
\end{block}

\begin{block}{vulture}
\footnotesize
Código não utilizado\\
\texttt{pip install vulture}\\
\texttt{vulture .}
\end{block}
\end{column}
\end{columns}

\vspace{0.5cm}
\begin{alertblock}{Exemplo Prático - Aula de Hoje}
\footnotesize
\faCode \, Vamos analisar código real usando essas ferramentas! \\
\faGithub \, Códigos \href{https://github.com/fmarquesfilho/bpp-2025-2/blob/main/src/s4/}{aqui}.
\end{alertblock}
\end{frame}

\begin{frame}[fragile]
\frametitle{Exemplo: Saída do pylint}
\begin{block}{Resultado da Análise}
\scriptsize
\begin{verbatim}
user_service.py:45:0: R0903: Too few public methods (1/2) (too-few-public-methods)
user_service.py:50:4: R0913: Too many arguments (6/5) (too-many-arguments)
user_service.py:50:4: R0915: Too many statements (55/50) (too-many-statements)
auth.py:23:4: W0612: Unused variable 'temp' (unused-variable)
utils.py:10:0: C0103: Function name "doStuff" doesn't conform to snake_case...
\end{verbatim}
\end{block}

\begin{block}{Code Smells Detectados}
\footnotesize
\begin{itemize}
    \item \textcolor{cleanred}{Too many arguments} → Long Parameter List
    \item \textcolor{cleanred}{Too many statements} → Long Method
    \item \textcolor{cleanred}{Unused variable} → Dead Code
    \item \textcolor{cleanred}{Bad naming} → Poor Naming
\end{itemize}
\end{block}
\end{frame}

\section{Princípios SOLID}

\begin{frame}
\frametitle{Princípios SOLID}
\begin{center}
\begin{tabular}{lll}
\textbf{S} & SRP & Single Responsibility Principle \\
\textbf{O} & OCP & Open/Closed Principle \\
\textbf{L} & LSP & Liskov Substitution Principle \\
\textbf{I} & ISP & Interface Segregation Principle \\
\textbf{D} & DIP & Dependency Inversion Principle \\
\end{tabular}
\end{center}

\vspace{0.5cm}
\begin{block}{Objetivo}
\footnotesize
\vspace{0.05cm}
Criar software mais flexível, compreensível e sustentável através de princípios de design orientado a objetos.
\end{block}
\end{frame}

\begin{frame}[fragile]
\frametitle{SRP - Single Responsibility Principle}
\begin{block}{\textcolor{cleanred}{\faTimes} Violação - Classe com múltiplas responsabilidades}
\scriptsize
\begin{lstlisting}
class UserManager:
    def create_user(self, user_data):
        # Valida dados + Salva no banco + Envia email + Gera log
        pass
        
    def generate_report(self):
        # Busca dados + Formata + Salva PDF + Envia email
        pass
\end{lstlisting}
\end{block}

\vspace{0.3cm}
\begin{alertblock}{Exemplos}
\footnotesize
\faGithub \, Códigos \href{https://github.com/fmarquesfilho/bpp-2025-2/blob/main/src/s4/}{aqui}.
\end{alertblock}
\end{frame}

\begin{frame}[fragile]
\frametitle{SRP - Single Responsibility Principle}
\begin{block}{\textcolor{cleangreen}{\faCheck} Solução - Responsabilidades separadas}
\scriptsize
\begin{lstlisting}
class UserValidator:
    def validate(self, user_data): pass

class UserRepository:
    def save(self, user_data): pass

class EmailService:
    def send_welcome_email(self, email): pass

class Logger:
    def log_user_creation(self, user_data): pass

# Cada classe tem uma única responsabilidade
\end{lstlisting}
\end{block}
\end{frame}

\begin{frame}[fragile]
\frametitle{OCP - Open/Closed Principle}
\begin{block}{\textcolor{cleanred}{\faTimes} Violação - Modificação para extensão}
\scriptsize
\begin{lstlisting}
class PaymentProcessor:
    def process_payment(self, payment_type, amount):
        if payment_type == "credit_card":
            # Processa cartao
        elif payment_type == "paypal":
            # Processa paypal
        elif payment_type == "pix":
            # Processa pix
        # Novo método -> modificar classe existente
\end{lstlisting}
\end{block}
\end{frame}

\begin{frame}[fragile]
\frametitle{OCP - Open/Closed Principle}
\begin{block}{\textcolor{cleangreen}{\faCheck} Solução - Extensão sem modificação}
\scriptsize
\begin{lstlisting}
from abc import ABC, abstractmethod

class PaymentMethod(ABC):
    @abstractmethod
    def process(self, amount): pass

class CreditCardPayment(PaymentMethod):
    def process(self, amount): pass

class PaymentProcessor:
    def process_payment(self, payment_method, amount):
        payment_method.process(amount)
\end{lstlisting}
\end{block}
\end{frame}

\section{Aplicação Prática}

\begin{frame}
\frametitle{Processo de Identificação e Refatoração}
\begin{center}
\begin{tikzpicture}[node distance=1.5cm, scale=0.7]
    \node[rectangle, draw, fill=cleanred!20, text width=2cm, align=center, font=\scriptsize] (problem) {Code Smell\\Identificado};
    \node[rectangle, draw, fill=yellow!20, text width=2cm, align=center, font=\scriptsize, right=of problem] (analyze) {Análise\\SOLID};
    \node[rectangle, draw, fill=cleangreen!20, text width=2cm, align=center, font=\scriptsize, right=of analyze] (solution) {Refatoração\\Aplicada};
    
    \draw[->, thick] (problem) -- (analyze);
    \draw[->, thick] (analyze) -- (solution);
\end{tikzpicture}
\end{center}

\vspace{0.5cm}
\begin{columns}
\begin{column}{0.5\textwidth}
\begin{block}{Processo}
\footnotesize
\begin{enumerate}
    \item \textbf{Identificar} code smells
    \item \textbf{Analisar} princípio SOLID violado
    \item \textbf{Aplicar} refatoração apropriada
    \item \textbf{Validar} melhoria na qualidade
\end{enumerate}
\end{block}
\end{column}

\begin{column}{0.5\textwidth}
\begin{alertblock}{Dica}
\vspace{0.05cm}
\footnotesize
Priorize refatorações baseado no impacto e frequência de mudanças na área do código.
\end{alertblock}
\end{column}
\end{columns}
\end{frame}

\begin{frame}
\frametitle{Exercícios Práticos - Aula de Hoje}
\begin{block}{Tarefa 1: Instalação das Ferramentas}
\footnotesize
\begin{enumerate}
    \item Instalar as ferramentas: \texttt{pip install pylint flake8 radon vulture}
    \item Verificar instalação: \texttt{pylint --version}
    \item Clonar repositório: \texttt{git clone https://github.com/fmarquesfilho/bpp-2025-2}
\end{enumerate}
\end{block}

\begin{block}{Tarefa 2: Análise com pylint}
\footnotesize
Analisar o arquivo \texttt{src/s4/problema1.py} e identificar:
\begin{itemize}
    \item Quais code smells foram detectados?
    \item Qual a pontuação do código?
    \item Quais princípios SOLID estão sendo violados?
\end{itemize}
\end{block}

\begin{block}{Tarefa 3: Métricas de Complexidade}
\footnotesize
Usar radon para medir complexidade: \texttt{radon cc src/s4 -a}
\end{block}
\end{frame}

\section{Conclusão}

\begin{frame}
\frametitle{Próximos Passos}
\begin{enumerate}
    \item \textbf{Prática}: Aplicar análise estática no seu projeto
    \item \textbf{Identificar}: Code smells no código atual
    \item \textbf{Refatorar}: Aplicar princípios SOLID
    \item \textbf{Documentar}: Registrar as melhorias
\end{enumerate}

\vspace{0.5cm}
\begin{block}{Ferramentas Recomendadas}
\footnotesize
\begin{itemize}
    \item \textbf{pylint}: Análise completa
    \item \textbf{radon}: Métricas de complexidade
    \item \textbf{vulture}: Dead code detection
    \item \textbf{flake8}: Estilo e boas práticas
\end{itemize}
\end{block}
\end{frame}

\begin{frame}
\frametitle{Referências}
\footnotesize
\begin{itemize}
    \item Fowler, Martin. \textbf{Refactoring: Improving the Design of Existing Code}
    \item Martin, Robert C. \textbf{Clean Code: A Handbook of Agile Software Craftsmanship}
    \item Catálogo de Code Smells: \url{https://luzkan.github.io/smells/}
    \item SOLID Principles: \url{https://medium.com/backticks-tildes/the-s-o-l-i-d-principles-in-pictures-b34ce2f1e898}
    \item Repositório do curso: \url{https://github.com/fmarquesfilho/bpp-2025-2}
\end{itemize}

\vspace{0.5cm}
\begin{center}
\Large
Dúvidas?
\end{center}
\end{frame}

\end{document}