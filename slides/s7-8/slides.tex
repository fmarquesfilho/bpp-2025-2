\documentclass[aspectratio=169,11pt]{beamer}

% Tema Metropolis
\usetheme{metropolis}

% Pacotes
\usepackage[utf8]{inputenc}
\usepackage[T1]{fontenc}
\usepackage[brazil]{babel}
\usepackage{listings}
\usepackage{xcolor}
\usepackage{hyperref}
\usepackage{tikz}
\usepackage{fontawesome5}
\usepackage{multicol}

% Bibliotecas TikZ
\usetikzlibrary{shapes.geometric,arrows,positioning,shadows}

% Configuração de cores
\definecolor{darkblue}{RGB}{0,51,102}
\definecolor{lightgray}{RGB}{240,240,240}
\definecolor{codegreen}{RGB}{0,128,0}
\definecolor{codepurple}{RGB}{153,0,153}
\definecolor{backcolour}{RGB}{248,248,248}

% Configuração de código
\lstdefinestyle{mystyle}{
    backgroundcolor=\color{backcolour},
    commentstyle=\color{codegreen},
    keywordstyle=\color{blue}\bfseries,
    numberstyle=\tiny\color{gray},
    stringstyle=\color{codepurple},
    basicstyle=\ttfamily\footnotesize,
    breakatwhitespace=false,
    breaklines=true,
    captionpos=b,
    keepspaces=true,
    numbers=left,
    numbersep=5pt,
    showspaces=false,
    showstringspaces=false,
    showtabs=false,
    tabsize=2,
    frame=single
}

% Configurações específicas
\lstset{style=mystyle}
\lstdefinestyle{pythoncode}{style=mystyle,language=Python}
\lstdefinestyle{yamlcode}{style=mystyle,language=bash}

% Informações do documento
\title{Mineração de Repositórios para Identificação de Code Smells}
\subtitle{Boas Práticas de Programação - BPP 2025.2}
\author{Prof. Fernando Marques Filho}
\date{10 e 17 de Outubro de 2025}
\institute{Universidade Federal do Rio Grande do Norte}

\begin{document}

% Slide 1: Título
\begin{frame}
\titlepage
\end{frame}

% Slide 2: Agenda
\begin{frame}{Agenda}
\tableofcontents
\end{frame}

% Seção 1
\section{Introdução à Mineração de Repositórios}

\begin{frame}{O que é Mineração de Repositórios?}
\begin{block}{Definição}
\textbf{Mining Software Repositories (MSR)} é a análise sistemática de dados históricos de repositórios para extrair padrões e insights sobre o desenvolvimento de software.
\end{block}

\begin{columns}[T]
\column{0.5\textwidth}
\textbf{O que analisamos?}
\begin{itemize}
    \item \faIcon{code-branch} Commits e mudanças
    \item \faIcon{bug} Issues e bugs
    \item \faIcon{comments} Discussões
    \item \faIcon{history} Histórico de mudanças
\end{itemize}

\column{0.5\textwidth}
\textbf{Para que serve?}
\begin{itemize}
    \item \faIcon{search} Identificar code smells
    \item \faIcon{chart-line} Prever problemas
    \item \faIcon{star} Avaliar qualidade
    \item \faIcon{lightbulb} Tomar decisões
\end{itemize}
\end{columns}
\end{frame}

\begin{frame}{Análise Estática vs Mineração de Repositórios}
\begin{columns}[T]
\column{0.48\textwidth}
\begin{block}{\faIcon{code} Análise Estática}
\textbf{Snapshot atual}
\begin{itemize}
    \item Examina código atual
    \item Problemas estruturais
    \item Code smells estáticos
    \item Violações de padrões
\end{itemize}
\end{block}

\column{0.48\textwidth}
\begin{block}{\faIcon{github} Mineração}
\textbf{Evolução temporal}
\begin{itemize}
    \item Analisa histórico
    \item Padrões de mudança
    \item Smells temporais
    \item Correlações
\end{itemize}
\end{block}
\end{columns}

\begin{center}
\textbf{Combinamos ambas para visão completa!}
\end{center}
\end{frame}

\begin{frame}{Por que Minerar Repositórios?}
\begin{enumerate}
    \item \textbf{Identificar Hotspots}
    \begin{itemize}
        \item Arquivos que mudam frequentemente
        \item Alta complexidade + muitas mudanças = risco
    \end{itemize}
    
    \item \textbf{Detectar Code Smells Temporais}
    \begin{itemize}
        \item Shotgun Surgery
        \item Divergent Change
        \item God Class evolutivo
    \end{itemize}
    
    \item \textbf{Priorizar Refatorações}
    \begin{itemize}
        \item Focar onde mais importa
    \end{itemize}
    
    \item \textbf{Entender Conhecimento}
    \begin{itemize}
        \item Especialistas por área
        \item Riscos de concentração
    \end{itemize}
\end{enumerate}
\end{frame}

% Seção 2
\section{Ferramentas para MSR}

\begin{frame}{Ferramentas Gratuitas - Visão Geral}
\begin{table}[]
\centering
\small
\begin{tabular}{|l|l|l|}
\hline
\textbf{Ferramenta} & \textbf{Linguagens} & \textbf{Foco} \\ \hline
SonarCloud & 30+ & Análise completa \\ \hline
CodeClimate & Python, JS, Ruby & Manutenibilidade \\ \hline
PyDriller & Python & Mineração Git \\ \hline
CodeScene & Múltiplas & Análise comportamental \\ \hline
\end{tabular}
\end{table}

\begin{alertblock}{\faIcon{info-circle} Importante}
Todas gratuitas para projetos acadêmicos e open source!
\end{alertblock}
\end{frame}

\begin{frame}{SonarCloud - Análise Contínua}
\begin{columns}[T]
\column{0.5\textwidth}
\textbf{Características:}
\begin{itemize}
    \item \url{https://sonarcloud.io}
    \item Gratuito para projetos públicos
    \item Integração nativa com GitHub
    \item 30+ linguagens suportadas
\end{itemize}

\column{0.5\textwidth}
\textbf{O que detecta:}
\begin{itemize}
    \item \faIcon{bug} Bugs
    \item \faIcon{shield-alt} Vulnerabilidades
    \item \faIcon{exclamation-triangle} Code Smells
    \item \faIcon{copy} Duplicação
\end{itemize}
\end{columns}

\begin{block}{Métricas Principais}
Reliability | Security | Maintainability | Coverage
\end{block}
\end{frame}

\begin{frame}[fragile]{PyDriller - Mineração com Python}
\begin{block}{O que é?}
Biblioteca Python para análise automatizada de repositórios Git.
\end{block}

\textbf{Instalação:}
\begin{lstlisting}[style=yamlcode,basicstyle=\ttfamily\footnotesize]
pip install pydriller
\end{lstlisting}

\begin{columns}[T]
\column{0.5\textwidth}
\textbf{Recursos:}
\begin{itemize}
    \item Análise de commits
    \item Estatísticas de arquivos
    \item Métricas de desenvolvedores
    \item Complexidade temporal
\end{itemize}

\column{0.5\textwidth}
\textbf{Vantagens:}
\begin{itemize}
    \item Fácil de usar
    \item Flexível
    \item Integração Python
\end{itemize}
\end{columns}
\end{frame}

\begin{frame}[fragile]{PyDriller - Exemplo Prático}
\begin{lstlisting}[style=pythoncode,basicstyle=\ttfamily\scriptsize]
from pydriller import Repository
from collections import Counter

# Analisar arquivos mais modificados
file_changes = Counter()

for commit in Repository(
    "https://github.com/user/repo", 
    since="2024-01-01"
).traverse_commits():
    for file in commit.modified_files:
        file_changes[file.filename] += 1

# Top 10 arquivos mais modificados
for file, count in file_changes.most_common(10):
    print(f"{file}: {count} mudancas")
\end{lstlisting}
\end{frame}

\begin{frame}{CodeScene - Análise Comportamental}
\begin{columns}[T]
\column{0.5\textwidth}
\textbf{O que é?}
\begin{itemize}
    \item Análise comportamental de código
    \item Combina código + histórico
    \item Visualizações avançadas
    \item Free tier (3 repositórios)
\end{itemize}

\column{0.5\textwidth}
\textbf{Recursos:}
\begin{itemize}
    \item Hotspots
    \item Code Health
    \item Knowledge Map
    \item Change Coupling
\end{itemize}

\begin{alertblock}{Diferencial}
Identifica problemas que análise estática não vê!
\end{alertblock}
\end{columns}
\end{frame}

% Seção 3
\section{MSR na Prática}

\begin{frame}[fragile]{Identificando Hotspots}
\begin{lstlisting}[style=pythoncode,basicstyle=\ttfamily\scriptsize]
from pydriller import Repository

# Configurar analise
repo_url = "https://github.com/user/repo"
hotspots = {}

for commit in Repository(repo_url).traverse_commits():
    for file in commit.modified_files:
        filename = file.filename
        if filename not in hotspots:
            hotspots[filename] = 0
        hotspots[filename] += 1

# Ordenar por frequencia
sorted_hotspots = sorted(hotspots.items(), 
                        key=lambda x: x[1], 
                        reverse=True)
\end{lstlisting}
\end{frame}

\begin{frame}{Interpretando Hotspots}
\begin{alertblock}{\faIcon{exclamation-triangle} Hotspots Problemáticos}
\begin{itemize}
    \item \textbf{God Class}: Muitas responsabilidades
    \item \textbf{Shotgun Surgery}: Mudanças espalhadas
    \item \textbf{Feature Envy}: Comportamento em classe errada
    \item \textbf{Instabilidade}: Muitas mudanças = muitos bugs
\end{itemize}
\end{alertblock}

\begin{block}{\faIcon{lightbulb} Priorização}
\begin{center}
Alto churn + Alta complexidade = \textbf{Prioridade máxima!}
\end{center}
\end{block}
\end{frame}

\begin{frame}[fragile]{Analisando Desenvolvedores}
\begin{lstlisting}[style=pythoncode,basicstyle=\ttfamily\scriptsize]
from collections import Counter
from pydriller import Repository

# Mapear conhecimento da equipe
dev_expertise = {}

for commit in Repository(".").traverse_commits():
    author = commit.author.name
    for file in commit.modified_files:
        if file.filename.endswith('.py'):
            if author not in dev_expertise:
                dev_expertise[author] = set()
            dev_expertise[author].add(file.filename)

# Exibir especialistas por arquivo
for dev, files in dev_expertise.items():
    print(f"{dev}: {len(files)} arquivos")
\end{lstlisting}
\end{frame}

\begin{frame}{Métricas de Qualidade - Referência}
\begin{table}[]
\centering
\scriptsize
\begin{tabular}{|l|l|l|l|}
\hline
\textbf{Problema} & \textbf{Métrica} & \textbf{Ferramenta} & \textbf{Limite} \\ \hline
Long Method & LOC & SonarCloud & > 50 \\ \hline
God Class & LOC + métodos & CodeClimate & > 500 \\ \hline
Duplicação & \% duplicada & SonarCloud & > 3\% \\ \hline
Complexidade & CC & Radon & > 10 \\ \hline
Shotgun Surgery & Frequência & PyDriller & Alta \\ \hline
\end{tabular}
\end{table}

\begin{block}{\faIcon{graduation-cap} Dica Acadêmica}
Use como guia, não como regra absoluta. Contexto importa!
\end{block}
\end{frame}

% Seção 4
\section{Automação com CI/CD}

\begin{frame}[fragile]{GitHub Actions - Análise Automatizada}
\begin{lstlisting}[style=yamlcode,basicstyle=\ttfamily\scriptsize]
name: Code Quality Analysis
on: [push, pull_request]

jobs:
  quality:
    runs-on: ubuntu-latest
    steps:
    - uses: actions/checkout@v3
    - uses: actions/setup-python@v4
      with:
        python-version: '3.x'
    
    - name: Install dependencies
      run: pip install pydriller pylint radon
    
    - name: Run analysis
      run: |
        python analysis_script.py
        pylint **/*.py
\end{lstlisting}
\end{frame}

\begin{frame}{Benefícios da Automação}
\begin{columns}[T]
\column{0.5\textwidth}
\textbf{Para Desenvolvedores:}
\begin{itemize}
    \item \faIcon{bolt} Feedback imediato
    \item \faIcon{shield-alt} Prevenção de problemas
    \item \faIcon{chart-line} Acompanhamento contínuo
    \item \faIcon{graduation-cap} Aprendizado contínuo
\end{itemize}

\column{0.5\textwidth}
\textbf{Para o Projeto:}
\begin{itemize}
    \item \faIcon{star} Qualidade consistente
    \item \faIcon{clock} Redução de revisões
    \item \faIcon{users} Padrão uniforme
    \item \faIcon{history} Histórico de métricas
\end{itemize}
\end{columns}
\end{frame}

% Seção 5
\section{Prática em Sala}

\begin{frame}{Exercício 1: Configuração Rápida (20 min)}
\begin{block}{\faIcon{rocket} Configurar Ferramentas}
\begin{enumerate}
    \item Acesse \url{https://sonarcloud.io}
    \item Login com GitHub
    \item Analyze new project
    \item Selecione repositório da disciplina
    \item Execute análise inicial
    \item Anote 3 code smells principais
\end{enumerate}
\end{block}

\begin{alertblock}{\faIcon{exclamation-triangle} Importante}
Repositório deve ser público para SonarCloud gratuito!
\end{alertblock}
\end{frame}

\begin{frame}{Exercício 2: Análise com PyDriller (25 min)}
\begin{block}{\faIcon{search} Identificar Hotspots}
\begin{enumerate}
    \item Clone seu repositório BPP
    \item Instale PyDriller: \texttt{pip install pydriller}
    \item Execute script de análise
    \item Identifique top 5 arquivos mais modificados
    \item Documente no README.md
\end{enumerate}
\end{block}

\begin{exampleblock}{\faIcon{code} Script Base}
\scriptsize
\texttt{from pydriller import Repository} \\
\texttt{for commit in Repository(".").traverse\_commits():} \\
\texttt{\ \ \ \ for file in commit.modified\_files:} \\
\texttt{\ \ \ \ \ \ \ \ print(f"{file.filename} - {commit.author}")}
\end{exampleblock}
\end{frame}

\begin{frame}{Exercício 3: Análise Comparativa (Para Casa)}
\begin{block}{\faIcon{chart-bar} Comparação entre Projetos}
Escolha 2 repositórios populares e compare:

\begin{columns}[T]
\column{0.5\textwidth}
\textbf{Qualidade:}
\begin{itemize}
    \item Technical debt
    \item Code smells
    \item Cobertura de testes
    \item Duplicação
\end{itemize}

\column{0.5\textwidth}
\textbf{Evolução:}
\begin{itemize}
    \item Hotspots
    \item Frequência de commits
    \item Tamanho das mudanças
    \item Distribuição de contribuições
\end{itemize}
\end{columns}
\end{block}
\end{frame}

% Seção 6
\section{Integração com BPP}

\begin{frame}{Aplicação no Projeto BPP}
\begin{block}{Critérios de Avaliação - U3}
\begin{itemize}
    \item \textbf{Code Smells Identificados} (20\%)
    \item \textbf{Refatorações Aplicadas} (20\%) 
    \item \textbf{Qualidade do Código} (30\%)
    \item \textbf{Documentação da Análise} (30\%)
\end{itemize}
\end{block}

\begin{exampleblock}{\faIcon{check-circle} Checklist Obrigatório}
\begin{itemize}
    \item SonarCloud configurado
    \item Hotspots identificados
    \item Code smells documentados
    \item Refatorações aplicadas
    \item Screenshots das métricas
\end{itemize}
\end{exampleblock}
\end{frame}

\begin{frame}{Estrutura da Documentação}
\begin{block}{README.md - Seção "Análise de Qualidade"}
\begin{enumerate}
    \item \textbf{Ferramentas Utilizadas}
    \item \textbf{Code Smells Identificados}
    \item \textbf{Hotspots e Métricas}
    \item \textbf{Refatorações Aplicadas}
    \item \textbf{Evolução das Métricas}
\end{enumerate}
\end{block}

\begin{alertblock}{\faIcon{trophy} Dica Extra}
Evolução positiva será valorizada! Mostre melhorias ao longo do tempo.
\end{alertblock}
\end{frame}

% Seção 7
\section{Conclusão e Próximos Passos}

\begin{frame}{Resumo dos Aprendizados}
\begin{block}{Principais Conceitos}
\begin{enumerate}
    \item MSR complementa análise estática
    \item Hotspots indicam problemas reais
    \item Ferramentas gratuitas são poderosas
    \item Automação é essencial para qualidade
    \item Métricas guiam decisões de refatoração
\end{enumerate}
\end{block}

\begin{columns}[T]
\column{0.5\textwidth}
\textbf{Ferramentas-Chave:}
\begin{itemize}
    \item SonarCloud
    \item PyDriller  
    \item GitHub Actions
\end{itemize}

\column{0.5\textwidth}
\textbf{Próximos Tópicos:}
\begin{itemize}
    \item Técnicas de Refatoração
    \item Padrões de Projeto
    \item Testes Automatizados
\end{itemize}
\end{columns}
\end{frame}

\begin{frame}{Próxima Aula: Refatoração Baseada em Dados}
\begin{block}{17 de Outubro - Preparação}
\begin{itemize}
    \item Traga análise do seu projeto
    \item Liste code smells identificados
    \item Prepare perguntas específicas
    \item Tenha IDE cloud configurada
\end{itemize}
\end{block}

\begin{exampleblock}{\faIcon{cloud} IDEs na Nuvem Recomendadas}
\begin{itemize}
    \item GitHub Codespaces
    \item GitPod
    \item Replit
\end{itemize}
\end{exampleblock}
\end{frame}

\begin{frame}{Recursos e Links Úteis}
\begin{columns}[T]
\column{0.5\textwidth}
\textbf{Ferramentas:}
\begin{itemize}
    \item SonarCloud: \url{sonarcloud.io}
    \item PyDriller: \url{pydriller.readthedocs.io}
    \item CodeScene: \url{codescene.com}
\end{itemize}

\textbf{Documentação:}
\begin{itemize}
    \item Repositório BPP: \url{github.com/fmarquesfilho/bpp-2025-2}
\end{itemize}

\column{0.5\textwidth}
\textbf{Comunidade:}
\begin{itemize}
    \item Discord: \url{discord.gg/bbMFJBQRT8}
\end{itemize}

\textbf{Referências:}
\begin{itemize}
    \item Awesome MSR
    \item Catálogo de Code Smells
\end{itemize}
\end{columns}
\end{frame}

\begin{frame}[standout]
\begin{center}
\Huge
\textbf{Dúvidas?}

\vspace{1cm}

\Large
\faIcon{github} \texttt{fmarquesfilho/bpp-2025-2}

\vspace{0.5cm}

\faIcon{discord} \texttt{discord.gg/bbMFJBQRT8}

\vspace{1cm}

\normalsize
\textbf{Bom trabalho!}
\end{center}
\end{frame}

\end{document}